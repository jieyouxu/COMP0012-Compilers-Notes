\subsection{Intermediate Representation}

\begin{definition}[Intermediate Representation (IR)]
    \textit{Intermediate Representation} (IR) acts as an adapter between the \textit{source language} and the \textit{target language}. This helps to decouple the compiler's front-end and back-end, and allows optimization to be source language-agnostic.
    
    \begin{figure}[H]
        \centering
        \begin{tikzpicture}
            \node [block] (ir) {Intermediate Representation (IR)};
            \node [block, above left=2em of ir] (src-cpp) {C++};
            \node [block, below left=2em of ir] (src-c) {C};
            \node [block, above=1em of src-cpp] (src-java) {Java};
            \node [block, below=1em of src-c] (src-js) {JavaScript};
            \node [block, above right=2em of ir] (dest-mips) {MIPS};
            \node [block, below right=2em of ir] (dest-darwin) {Darwin};
            \node [block, above=1em of dest-mips] (dest-x86) {x86};
            \node [block, below=1em of dest-darwin] (dest-pentium) {Pentium};
            \draw [->] (src-cpp) -- (ir);
            \draw [->] (src-c) -- (ir);
            \draw [->] (src-java) -- (ir);
            \draw [->] (src-js) -- (ir);
            \draw [->] (ir) -- (dest-mips);
            \draw [->] (ir) -- (dest-darwin);
            \draw [->] (ir) -- (dest-x86);
            \draw [->] (ir) -- (dest-pentium);
            \draw [->] (ir) edge[loop above] node [midway, above] {Optimization} (ir);
        \end{tikzpicture}
        \caption{Intermediate representation between source language and destination language.}
        \label{fig:intermediate-representation}
    \end{figure}
\end{definition}

\begin{definition}[High-level IR vs Low-level IR]
    \textit{High-level} IR and \textit{low-level} IR differs in the amount of source information they retain.
    \begin{figure}[H]
        \centering
        \begin{tabularx}{\textwidth}{@{} X X @{}}
            \toprule
            High-level IR & Low-level IR \\
            \midrule
            Retains more source code information. Allows compiler to extract source code intention more easily. & Retains less source code information. Allows compiler to generate code for hardware platform more easily. \\
            \bottomrule
        \end{tabularx}
        \caption{High-level IR vs Low-level IR}
        \label{fig:high-level-ir-vs-low-level-ir}
    \end{figure}
\end{definition}

\begin{definition}[Properties of Good IR]
    A \textit{good IR} has the properties
    \begin{itemize}
        \item Accuracy.
        \item Decoupled from source and target language.
        \item Easy to produce.
        \item Easy to translate.
        \item Clear, unambiguous language constructs.
        \item No convenience features.
    \end{itemize}
\end{definition}

\begin{definition}[IR Categories]
    There exists three major categories of IR
    \begin{figure}[H]
        \centering
        \begin{tabularx}{\textwidth}{@{} X X X @{}}
            \toprule
            Structural IR & Hybrid IR & Linear IR \\
            \midrule
            Graphically structured & Both graphical and linear code & Pseudo-machine code \\
            Used heavily in transpilers and verification tools & & Different abstraction levels \\
            Usually large & & Easier to rearrange \\
            \midrule
            e.g.: Trees, DAGs & e.g. Control-Flow Graph & e.g.: Three-Address Code, Stack Machine Code \\
            \bottomrule
        \end{tabularx}
        \caption{Categories of Intermediate Representation}
        \label{fig:ir-categories}
    \end{figure}
\end{definition}

\subsubsection{Linear Intermediate Representation}

\begin{definition}[Linear IR]
    A \textit{Linear Intermediate Representation} (LIR) is an instruction set for an abstract machine.
    
    \begin{itemize}[topsep=4pt]
        \item Intermediate between Decorated AST and Machine Code.
        \item Allows language-agnostic optimizations.
    \end{itemize}
    
    It is ideal for optimization since
    \begin{itemize}[topsep=4pt]
        \item Linear AST has a \textit{narrow interface} of limited instructions.
        \item This allows optimizations, re-targeting and translating to become easier.
    \end{itemize}
    
    Examples include
    \begin{itemize}
        \item GCC Gimple
        \item LLVM IR
        \item Java Bytecode
    \end{itemize}
\end{definition}

\begin{definition}[Abstract Machines]
    LIRs are instructions for \textit{abstract machines}; two types of abstract machines are
    \begin{enumerate}
        \item \textbf{Register Machines}.
        \item \textbf{Stack Machines}.
    \end{enumerate}
\end{definition}

\begin{definition}[Register Machines]
    A \textit{register-based abstract machine} uses \textit{identifier-based temporary variables} (which later gets mapped to real registers during register-selection phase).
\end{definition}

\begin{example}
    A very well known example of register machine is MIPS, a corresponding register-based IR is \textit{Three-Address Code} (TAC).
\end{example}

\begin{definition}[Stack Machines]
    A \textit{stack machine} performs all operations based on \textit{stack}. It has two memory regions
    \begin{enumerate}
        \item \textbf{Instruction Memory}: which contains IR code.
        \item \textbf{Data Memory}: which stores values, referenced via id or address.
    \end{enumerate}
\end{definition}

\begin{example}
    \textit{Java Bytecode} is a stack-based IR which runs on the \textit{Java Virtual Machine}, a stack-based abstract machine.
\end{example}

\subsubsection{Register Machines and Three-Address Code}

\begin{definition}[Three-Address Code]
    \textit{Three-Address Code} (IR) is a register-based IR which has at most three addresses per instruction. Each instruction has the form
    \begin{equation}
        \mathrm{OP}\ a_{1}\ a_{2}\ a_{3}
    \end{equation}
    
    Where $a_1, a_2, a_3$ are arguments to the operation. The number of operand registers may vary based on the arity of the operation.
    
    Note that compound expressions may have to be partitioned into small consecutive steps (unary, binary expressions) using temporary registers to hold intermediate values due to the instruction size constraint.
\end{definition}

\begin{definition}[Three-Address Code Statements]
    Let us define a custom-flavoured Three-Address Code (TAC).
    
    \begin{sidewaystable}[H]
        \centering
        \setlength\tabcolsep{2.5pt}
        \scriptsize
        \begin{tabularx}{\textwidth}{c c c c c c c c c c c}
            \toprule
            \multicolumn{3}{c}{Binary} & & & & \multicolumn{3}{c}{Control Flow} & \\
            Arithmetic 
                & Logical 
                & Comparisons 
                & Unary 
                & Copy 
                & Array Access 
                & Field Access
                & Labelling
                & Unconditional Jump
                & Conditional Jump
                & Function Calls
                \\
            \midrule
            \texttt{ADD} 
                & \texttt{AND} 
                & \texttt{EQ}  
                & \texttt{MINUS} 
                & \texttt{a = b} 
                & \texttt{a = b[i]} 
                & \texttt{a = b.f} 
                & \texttt{label L:}
                & \texttt{jump L}
                & \texttt{tjump a L}
                & \texttt{param p}
                \\
            \texttt{SUB} 
                & \texttt{OR}  
                & \texttt{NEQ} 
                & \texttt{NEG}   
                &
                & \texttt{a[i] = b}
                & \texttt{a.f = b} 
                &
                &
                & \texttt{fjump a L}
                & \texttt{call f}
                \\
            \texttt{MUL} 
                & \texttt{XOR} 
                & \texttt{LE}  
                & 
                & 
                &
                &
                &
                &
                &
                & \texttt{a = call f}
                \\
            \texttt{DIV} 
                &
                & \texttt{LEQ} 
                &                
                &                
                & 
                & 
                &
                &
                &
                & \texttt{return}
                \\
            \texttt{MOD} 
                &
                & \texttt{GE}
                & 
                & 
                &
                &
                &
                &
                &
                & \texttt{return y}
                \\
            \phantom{} 
                &
                & \texttt{GEQ}
                &
                &
                &
                &
                &
                &
                &
                &
                \\
            \bottomrule
        \end{tabularx}
        \caption{Three-Address Code Instructions}
        \label{fig:tac-instructions}
    \end{sidewaystable}
\end{definition}

\begin{definition}[TAC Operands]
    Each \textit{operand} in a TAC instruction can be
    \begin{enumerate}
        \item Program variables
        \item Constants
        \item Temporary Variables
    \end{enumerate}
\end{definition}

\begin{definition}[Temporary Variables]
    \textit{Temporary variables} are new locations which are used to store intermediate values.
\end{definition}

\subsubsection{Translation}

\begin{definition}[Translation]
    Translation between languages requires dealing with \textit{nested constructs}. This derives the need for \textit{inductive}, \textit{template-based} translation
    \begin{itemize}
        \item Start from AST root node.
        \item Define translation rules for each node type.
        \item Recursively apply templates down AST.
    \end{itemize}
\end{definition}

\begin{definition}[Translation Notation]
    If $e$ is a high-level construct, then
    \begin{itemize}
        \item $T[e]$ is the sequence of  low-level translated instructions.
        \item If $e$ is an \textit{expression}, then $e$ represents a \textit{value};
        \begin{itemize}
           \item $t = T[e]$ means the result of $T[e]$ is stored at $t$.
        \end{itemize}
        \item If $e$ is a \textit{variable}, then $t = T[e]$ is the \textit{copy instruction} $t = e$.
    \end{itemize}
\end{definition}

\begin{definition}[Translating Expressions]
    There are generic templates:
    \begin{enumerate}
        \item \textbf{Unary operation}.
        
            \begin{equation*}
                t = T[\mathtt{OP}\ e]
            \end{equation*}
            
            It is translated to
            
            \begin{figure}[H]
                \centering
                \begin{subfigure}{0.45\textwidth}
                    \centering
                    \begin{forest}
                        [$\mathtt{OP}$
                            [$e$]
                        ]
                    \end{forest}
                \end{subfigure}
                \begin{subfigure}{0.45\textwidth}
                    \centering
                    \begin{align*}
                        t_1 &= T[e] \\
                        t &= \mathtt{OP}\ t_1 \\ 
                    \end{align*}
                \end{subfigure}
            \end{figure}
        
        \item \textbf{Binary operation}.
        
            \begin{equation*}
                t = T[e_1 \mathtt{OP}\ e_2]
            \end{equation*}
            
            It is translated to
            
            \begin{figure}[H]
                \centering
                \begin{subfigure}{0.45\textwidth}
                    \centering
                    \begin{forest}
                        [$\mathtt{OP}$
                            [$e_1$]
                            [$e_2$]
                        ]
                    \end{forest}
                \end{subfigure}
                \begin{subfigure}{0.45\textwidth}
                    \centering
                    \begin{align*}
                        t_1 &= T[e_1] \\
                        t_2 &= T[e_2] \\
                        t &= t_1 \mathrm{OP}\ t_2 \\ 
                    \end{align*}
                \end{subfigure}
            \end{figure}
        
        \item \textbf{Short-circuiting} \texttt{OR}.
        
            \begin{equation*}
                t = T[e_1 \mathtt{OR}\ e_2]
            \end{equation*}
            
            It is translated to
            
            \begin{figure}[H]
                \centering
                \begin{subfigure}{0.45\textwidth}
                    \centering
                    \begin{forest}
                        [$\mathtt{OR}$
                            [$e_1$]
                            [$e_2$]
                        ]
                    \end{forest}
                \end{subfigure}
                \begin{subfigure}{0.45\textwidth}
                    \centering
                    \begin{align*}
                        &t = T[e_1] \\ 
                        &\mathtt{tjump}\ t\ \mathtt{END} \\
                        &t = T[e_2] \\
                        &\mathtt{label\ END} \\
                    \end{align*}
                \end{subfigure}
            \end{figure}
        
            Doing so allows us to avoid evaluating $e_2$ supposing $e_1$ is true since $\top \lor q \equiv t$. This is espeially beneficial if $e_2$ involves some expensive calculations or I/O operations.
        
        \item \textbf{Short-circuiting} \texttt{AND}.
        
            \begin{equation*}
                t = T[e_1 \mathtt{AND}\ e_2]
            \end{equation*}
            
            It is translated to
            
            \begin{figure}[H]
                \centering
                \begin{subfigure}{0.45\textwidth}
                    \centering
                    \begin{forest}
                        [$\mathtt{AND}$
                            [$e_1$]
                            [$e_2$]
                        ]
                    \end{forest}
                \end{subfigure}
                \begin{subfigure}{0.45\textwidth}
                    \centering
                    \begin{align*}
                        &t = T[e_1] \\ 
                        &\mathtt{fjump}\ t\ \mathrm{END} \\
                        &t = T[e_2] \\
                        &\mathtt{label\ END} \\
                    \end{align*}
                \end{subfigure}
            \end{figure}
            
            This can be done because $\bot \land q \equiv \bot$.
        
        \item \textbf{Array access}.
        
            \begin{equation*}
                t = T[v[e]]
            \end{equation*}
            
            It is translated to
            
            \begin{figure}[H]
                \centering
                \begin{subfigure}{0.45\textwidth}
                    \centering
                    \begin{forest}
                        [$\mathtt{array}$
                            [$v$]
                            [$e$]
                        ]
                    \end{forest}
                \end{subfigure}
                \begin{subfigure}{0.45\textwidth}
                    \centering
                    \begin{align*}
                        &t_1 = T[e] \\ 
                        &t = v[t_1] \\
                    \end{align*}
                \end{subfigure}
            \end{figure}
        
        \item \textbf{Field access}.
        
            \begin{equation*}
                t = T[e_1\mathtt{.}e_2]
            \end{equation*}
            
            It is translated to
            
            \begin{figure}[H]
                \centering
                \begin{subfigure}{0.45\textwidth}
                    \centering
                    \begin{forest}
                        [$\mathtt{.}$
                            [$e_1$]
                            [$e_2$]
                        ]
                    \end{forest}
                \end{subfigure}
                \begin{subfigure}{0.45\textwidth}
                    \centering
                    \begin{align*}
                        &t_1 = T[e_1] \\ 
                        &t_2 = T[e_2] \\
                        &t = t_1.t_2 \\
                    \end{align*}
                \end{subfigure}
            \end{figure}
        
        \item \textbf{Statements list}.
        
            \begin{equation*}
                T[s_1; s_2; \dots; s_n]
            \end{equation*}
            
            It is translated to
            
            \begin{figure}[H]
                \centering
                \begin{subfigure}{0.45\textwidth}
                    \centering
                    \begin{forest}
                        [$\mathtt{seq}$
                            [$s_1$]
                            [$s_2$]
                            [$\dots$]
                            [$s_n$]
                        ]
                    \end{forest}
                \end{subfigure}
                \begin{subfigure}{0.45\textwidth}
                    \centering
                    \begin{align*}
                        &T[s_1] \\
                        &T[s_2] \\
                        &\dots  \\
                        &T[s_n] \\
                    \end{align*}
                \end{subfigure}
            \end{figure}
        
        \item \textbf{Variable assignment}.
        
            \begin{equation*}
                T[v = e]
            \end{equation*}
            
            It is translated to
            
            \begin{figure}[H]
                \centering
                \begin{subfigure}{0.45\textwidth}
                    \centering
                    \begin{forest}
                        [$\mathtt{variable-assign}$
                            [$v$]
                            [$e$]
                        ]
                    \end{forest}
                \end{subfigure}
                \begin{subfigure}{0.45\textwidth}
                    \centering
                    \begin{align*}
                        &v = T[e] \\
                    \end{align*}
                \end{subfigure}
            \end{figure}
            
        \item \textbf{Array assignment}.
        
            \begin{equation*}
                T[v[e_1] = e_2]
            \end{equation*}
            
            It is translated to
            
            \begin{figure}[H]
                \centering
                \begin{subfigure}{0.45\textwidth}
                    \centering
                    \begin{forest}
                        [$\mathtt{array-assign}$
                            [$v$]
                            [$e_1$]
                            [$e_2$]
                        ]
                    \end{forest}
                \end{subfigure}
                \begin{subfigure}{0.45\textwidth}
                    \centering
                    \begin{align*}
                        t_1 &= T[e_1] \\
                        t_2 &= T[e2_2] \\
                        v[t_1] &= t_2 \\
                    \end{align*}
                \end{subfigure}
            \end{figure}
        
        \item \textbf{If-then}.
        
            \begin{equation*}
                T[\mathtt{if}\ (e)\ \mathtt{then}\ s]
            \end{equation*}
            
            It is translated to
            
            \begin{figure}[H]
                \centering
                \begin{subfigure}{0.45\textwidth}
                    \centering
                    \begin{forest}
                        [$\mathtt{if-then}$
                            [$e$]
                            [$s$]
                        ]
                    \end{forest}
                \end{subfigure}
                \begin{subfigure}{0.45\textwidth}
                    \centering
                    \begin{align*}
                        &t_1 = T[e] \\
                        &t_2 = \mathtt{NOT}\ t_1 \\
                        &\mathtt{tjump}\ t_2\ \mathtt{END} \\
                        &T[s] \\
                        &\mathtt{label\ END} \\
                    \end{align*}
                \end{subfigure}
            \end{figure}
        
        \item \textbf{If-then-else}.
        
            \begin{equation*}
                T[\mathtt{if}\ (e)\ \mathtt{then}\ s_1\ \mathtt{else}\ s_2]
            \end{equation*}
            
            It is translated to
            
            \begin{figure}[H]
                \centering
                \begin{subfigure}{0.45\textwidth}
                    \centering
                    \begin{forest}
                        [$\mathtt{if-then-else}$
                            [$e$]
                            [$s_1$]
                            [$s_2$]
                        ]
                    \end{forest}
                \end{subfigure}
                \begin{subfigure}{0.45\textwidth}
                    \centering
                    \begin{align*}
                        &t_1 = T[e] \\
                        &\mathtt{tjump}\ t_1\ \mathtt{LEFT} \\
                        &T[s_2] \\
                        &\mathtt{jump\ END} \\
                        &\mathtt{label\ LEFT} \\
                        &T[s_1] \\
                        &\mathtt{label\ END} \\
                    \end{align*}
                \end{subfigure}
            \end{figure}
        
        \item \textbf{While}.
        
            \begin{equation*}
                T[\mathtt{while}\ (e)\ \lbrace s \rbrace]
            \end{equation*}
            
            It is translated to
            
            \begin{figure}[H]
                \centering
                \begin{subfigure}{0.45\textwidth}
                    \centering
                    \begin{forest}
                        [$\mathtt{while}$
                            [$e$]
                            [$s$]
                        ]
                    \end{forest}
                \end{subfigure}
                \begin{subfigure}{0.45\textwidth}
                    \centering
                    \begin{align*}
                        &\mathtt{label\ TEST} \\
                        &t_1 = T[e] \\
                        &\mathtt{fjump}\ t_1\ \mathtt{END} \\
                        &T[s] \\
                        &\mathtt{jump\ TEST} \\
                        &\mathtt{label\ END} \\
                    \end{align*}
                \end{subfigure}
            \end{figure}
        
        \item \textbf{Function call}.
        
            \begin{equation*}
                T[f(e_1, e_2, \dots, e_n)]
            \end{equation*}
            
            It is translated to
            
            \begin{figure}[H]
                \centering
                \begin{subfigure}{0.45\textwidth}
                    \centering
                    \begin{forest}
                        [$\mathtt{function-call}$
                            [$e_1$]
                            [$e_2$]
                            [$\dots$]
                            [$e_n$]
                        ]
                    \end{forest}
                \end{subfigure}
                \begin{subfigure}{0.45\textwidth}
                    \centering
                    \begin{align*}
                        &\mathtt{param}\ T[e_1] \\
                        &\mathtt{param}\ T[e_2] \\
                        &\dots \\
                        &\mathtt{param}\ T[e_n] \\
                        &\mathtt{call}\ f \\
                    \end{align*}
                \end{subfigure}
            \end{figure}
        
        \item \textbf{Return statement}.
        
            \begin{equation*}
                T[\mathtt{return}\ e]
            \end{equation*}
            
            It is translated to
            
            \begin{figure}[H]
                \centering
                \begin{subfigure}{0.45\textwidth}
                    \centering
                    \begin{forest}
                        [$\mathtt{return}$
                            [$e$]
                        ]
                    \end{forest}
                \end{subfigure}
                \begin{subfigure}{0.45\textwidth}
                    \centering
                    \begin{align*}
                        &t = T[e] \\
                        &\mathtt{return}\ t;
                    \end{align*}
                \end{subfigure}
            \end{figure}
    \end{enumerate}
\end{definition}

\begin{definition}[TAC Recursive Translation]
    With the previously defined templates in mind, it is then possible to recursively apply the templates in order to translate nested constructs.
\end{definition}

\begin{example}
    Given the code
    \begin{verbatim}
if (c)
{
    if (d)
    {
        a = b
    }
}
print(x)
    \end{verbatim}
    
    Which generates the AST
    \begin{figure}[H]
        \begin{subfigure}{0.45\textwidth}
            \centering
            \begin{forest}
                [block
                    [if-then
                        [c]
                        [if-then
                            [d]
                            [assign
                                [a]
                                [b]
                            ]
                        ]
                    ]
                    [call
                        [print]
                        [x]
                    ]
                ]
            \end{forest}
        \end{subfigure}
        \begin{subfigure}{0.45\textwidth}
            \centering
            \begin{align*}
                &t_1 = T[c] \\
                &\mathtt{tjump}\ t_1\ \mathtt{OUTER\_TRUE} \\
                &\mathtt{param}\ x \\
                &\mathtt{call}\ print \\
                &\mathtt{jump}\ \mathtt{OUTER\_END} \\
                &\mathtt{label\ OUTER\_TRUE} \\
                &t_2 = T[d] \\
                &\mathtt{tjump}\ t_2\ \mathtt{INNER\_TRUE} \\
                &\mathtt{jump}\ \mathtt{INNER\_END} \\
                &\mathtt{label\ INNER\_TRUE} \\
                &t_3 = T[b] \\
                &t_4 = T[a] \\
                &t_4 = t_3 \\
                &\mathtt{label\ INNER\_END} \\
                &\mathtt{label\ OUTER\_END}
            \end{align*}
        \end{subfigure}
    \end{figure}
\end{example}

\subsubsection{Implementing Three-Address Code}

\begin{definition}[TAC Implementation]
    \textit{Three-Address Code} (TAC) can be implemented by
    \begin{itemize}
        \item \textbf{Quadruples}
        \item \textbf{Triples}
        \item \textbf{Indirect Triples}
    \end{itemize}
\end{definition}

\begin{definition}[Quadruples]
    Each instruction in \textit{Quadruples} consist of an operator, two arguments field, as well as an result field.
    \begin{itemize}
        \item \textit{Operator}: type of instruction.
        \item \textit{Arguments}: at most two, possibly empty.
        \item \textit{Result}: address to store computed result.
    \end{itemize}
    
    Note that \texttt{jump} instructions place label address in the result field.
    
    For example, the code fragment \texttt{a = -c + b} could be represented as
    \begin{figure}[H]
        \centering
        \begin{tabular}{@{} c c c c @{}}
            \toprule
            Operator & Argument 1 & Argument 2 & Result \\
            \midrule
            \texttt{UMINUS} & $c$   &       & $t_1$ \\
            \texttt{PLUS}   & $b$   & $t_1$ & $t_2$ \\
            \texttt{ASSIGN} & $t_2$ &       & $a$ \\
            \bottomrule
        \end{tabular}
        \caption{Example Quadruples}
        \label{fig:example-quadruples}
    \end{figure}
\end{definition}

\begin{definition}[Triples]
    In \textit{Triples}, results are used as temporaries.
    
    For example
    \begin{figure}[H]
        \centering
        \begin{tabular}{@{} c c c c @{}}
            \toprule
            Address & Operator & Argument 1 & Argument 2 \\
            \midrule
            20 & \texttt{UMINUS} & $c$   &        \\
            21 & \texttt{PLUS}   & $b$   & $[20]$ \\
            22 & \texttt{ASSIGN} & $a$   & $[21]$ \\
            \bottomrule
        \end{tabular}
        \caption{Example Triples}
        \label{fig:example-triples}
    \end{figure}
\end{definition}

\begin{remark}
    Note that Triples suffer from the problem that since optimization tends to need to move code around, operation addresses are not preserved and will need to be updated.
\end{remark}

\begin{definition}[Indirect Triples]
    \textit{Indirect Triples} improve upon Triples by adding an additional address-resolution (pointer) table which allows instruction to refer to "virtual" addresses instead of fixed absolute addresses.
    
    For example,
    \begin{figure}[H]
        \centering
        \begin{tabular}{@{} c c @{}}
            \toprule
            Pointer & Address \\
            \midrule
            $[20]$ & $45$ \\
            $[21]$ & $46$ \\
            $[22]$ & $47$ \\
            \bottomrule
        \end{tabular}
        \quad
        \begin{tabular}{@{} c c c c @{}}
            \toprule
            Address & Operator & Argument 1 & Argument 2 \\
            \midrule
            20 & \texttt{UMINUS} & $c$   &        \\
            21 & \texttt{PLUS}   & $b$   & $[20]$ \\
            22 & \texttt{ASSIGN} & $a$   & $[21]$ \\
            \bottomrule
        \end{tabular}
        \caption{Example Indirect Triples}
        \label{fig:example-indirect-triples}
    \end{figure}
\end{definition}

\begin{remark}
    Each type of TAC implementation has its own benefits and caveats.
    \begin{figure}[H]
        \centering
        \begin{tabularx}{\textwidth}{@{} X X X @{}}
            \toprule
            Quadruples & Triples & Indirect Triples \\
            \midrule
            Easier to optimize
            & Need to update references upon moving instructions
            & Need to reorder statements but no changes to references \\
            Could generate excessive temporaries 
            &
            & Saves space if temporaries are reused \\
            \bottomrule
        \end{tabularx}
        \caption{Benefits and issues of the different TAC implementations}
        \label{fig:tac-implemenations-evaluation}
    \end{figure}
\end{remark}

\subsubsection{Stack Machines}

\begin{definition}[Stack Machine]
    A \textit{Stack Machine} utilizes a \textit{stack} to perform operations which stores operands and results.
\end{definition}

\begin{definition}[Abstract Stack Machine]
    An \textit{Abstract Stack Machine} (ASM) has the properties
    \begin{itemize}
        \item Operands and results stored in stack.
        \item No need for register allocation.
        \item Detailed run-time organization can be delayed.
    \end{itemize}
\end{definition}

\begin{definition}[Benefits and Drawbacks of Stack-based IRs]
    Stack-Based IRs have the following benefits and drawbacks:
    \begin{itemize}
        \item Benefits
        \begin{itemize}
            \item Generate more compact IR.
            \item Simpler compiler because independent instructions.
            \item Simpler interpreter due to centralized memory access and less variation in instruction types.
        \end{itemize}
        \item Drawbacks
        \begin{itemize}
            \item More memory references are needed since variables need to be pushed back on to the stack.
            \item Difficult to factor out common sub-expressions.
            \item Instruction order tied to storage of temporary values making moving instructions around difficult, making optimization more difficult.
        \end{itemize}
    \end{itemize}
\end{definition}
