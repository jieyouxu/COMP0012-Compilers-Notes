\section{Introduction}

Notes from COMP0012 Lecture Slides.

\subsection{Compiler Importance}
\begin{itemize}
    \item Raise abstraction level.
    \item Bridge gap between human mind and machine execution.
    \item Increase maintainability of code.
\end{itemize}

\subsection{Compiler Goals}
\begin{itemize}
    \item Translate source code into \textit{equivalent} machine code.
    \item Translation efficiency.
    \item Useful error messages.
    \item Parallel compilation support.
    \item Program correctness verification.
\end{itemize}

\subsection{High-level Compiler Structure}
\begin{figure}[H]
    \centering
    \fbox{
    \begin{tikzpicture}[auto]
        \node [dashedblock] (n1) { Source Code };
        \node [block, below=1em of n1] (n2) { Analysis };
        \node [dashedblock, below=1em of n2] (n3) { Intermediate Representation };
        \node [block, below=1em of n3] (n4) { Synthesis };
        \node [dashedblock, below=1em of n4] (n5) { Target Code };
        
        \foreach \i/\j in {1/2, 2/3, 3/4, 4/5}{
            \draw[thick, ->] (n\i) -- (n\j);
        }
    \end{tikzpicture}
    }
    \caption{Compiler high-level overview}
    \label{fig:compiler-high-level-overview}
\end{figure}

\subsection{Compiler Front-end}
\begin{figure}[H]
    \centering
    \fbox{
    \begin{tikzpicture}[auto]
        \node [dashedblock] (n1) { Source ( \texttt{Stream<Char>} ) };
        \node [block, below=2em of n1] (n2) { Lexical Analysis (Lexer/Scanner) };
        \draw [thick, ->] (n1) -- (n2);
        \node [block, below=2em of n2] (n3) { Syntax Analysis (Parser) };
        \draw [thick, ->] (n2) -- (n3) node [midway] { Token Stream };
        \node [block, below=2em of n3] (n4) { Semantic Analysis };
        \draw [thick, ->] (n3) -- (n4) node [midway] { Abstract Syntax Tree (AST) };
        \node [phantomblock, below=2em of n4] (n5) { \textit{to back-end} $\cdots$ };
        \draw [thick, ->] (n4) -- (n5) node [midway] { Decorated Abstract Syntax Tree (DAST) };
    \end{tikzpicture}
    }
    \caption{Compiler front-end overview}
    \label{fig:compiler-front-end-overview}
\end{figure}

\subsection{Compiler Back-end}
\begin{figure}[H]
    \centering
    \fbox{
    \begin{tikzpicture}[auto]
        \node [phantomblock] (n1) { $\cdots$ \textit{from front-end} };
        \node [block, below=2em of n1] (n2) { Intermediate Code Generator };
        \draw [thick, ->] (n1) -- (n2) node [midway] { Decorated Abstract Syntax Tree (DAST) };
        \node [block, below=2em of n2] (n3) { Optimizer };
        \draw [thick, ->] (n2) -- (n3) node [midway] { Intermediate Representation (IR) };
        \node [phantomblock, right=2em of n3] (n4) { IR };
        \draw [thick, ->] (n3) to [out=300,in=60] (n4);
        \draw [thick, ->] (n4) to [out=300,in=60] (n3);
        \node [block, below=2em of n3] (n4) { Code Generator };
        \draw [thick, ->] (n3) -- (n4) node [midway] { IR };
        \node [dashedblock, below=2em of n4] (n5) { Machine Code };
        \draw [thick, ->] (n4) -- (n5);
    \end{tikzpicture}
    }
    \caption{Compiler back-end overview}
    \label{fig:compiler-back-end-overview}
\end{figure}

\subsection{Lexical Analysis}

To extract \textit{lexemes}\footnote{Note that \textit{lexeme} refers to a word in the source code, while a \textit{token} is a set of related lemexes such as \textit{identifiers}.} from source code:

\begin{figure}[H]
    \begin{subfigure}{0.45\textwidth}
        \centering
        \begin{minted}{c}
        int main(void)
        {
            return 0;
        }
        \end{minted}
        \caption{Source Code}
        \label{fig:source-code-to-lexemes-example-source-code}
    \end{subfigure}
    \begin{subfigure}{0.45\textwidth}
        \centering
        \begin{minted}{text}
        INT INDENTIFIER("MAIN") LPAREN
        VOID RPAREN LBRACE RETURN 
        INT_LITERAL(0) SEMI RBRACE
        \end{minted}
        \caption{Lexemes}
    \end{subfigure}
    \caption{Example of converting source code to lexemes}
    \label{fig:source-code-to-lexemes-example}
\end{figure}

\subsection{Syntax Analysis}

To recognize \textit{structure} of a phrase in the language, which produces an \textit{Abstract Syntax Tree} (AST).

\begin{figure}[H]
    \centering
    \begin{forest}
        [function-definition-statement
            [function-signature
                [return-type
                    [\texttt{int}]
                ]
                [identifier
                    [\texttt{main}]
                ]
                [argument-list
                    [\texttt{(}]
                    [argument
                        [\texttt{void}]
                    ]
                    [\texttt{)}]
                ]
            ]
            [block
                [\texttt{\{}]
                [statement-list
                    [statement
                        [return-statement
                            [\texttt{return}]
                        ]
                        [expression
                            [int-literal
                                [\texttt{0}]
                            ]
                        ]
                        [\texttt{;}]
                    ]
                ]
                [\texttt{\}}]
            ]
        ]
    \end{forest}
    \caption{Simple parse tree of Figure \ref{fig:source-code-to-lexemes-example-source-code}}
    \label{fig:simple-parse-tree-example}
\end{figure}

\subsection{Semantic Analysis}

Validate the source code to have acceptable \textit{meaning}. Note that while the \textit{syntax} may be valid, but the \textit{semantics} may not be.

\begin{figure}[H]
    \centering
    \begin{minted}{java}
        String a;
        if (a == 1)
            System.out.println(a);
    \end{minted}
    \caption{Syntactically correct but semantically wrong. Note that trying to use the 
        referential equality comparison operator \texttt{==} with two arguments of 
        different types in this case yields a \texttt{TypeError}. }
    \label{fig:my_label}
\end{figure}

\subsection{Error Handling and Symbol Table}

\begin{figure}[H]
    \centering
    \fbox{
    \begin{tikzpicture}[auto]
        \node [dashedblock] (n0) { Source Code };
        \node [block, below=2em of n0] (n1) { Lexical Analysis (Lexer) };
        \draw [thick, ->] (n0) -- (n1);
        \node [block, below=3em of n1] (n2) { Syntax Analysis (Parser) };
        \draw [thick, ->] (n1) -- (n2) node [midway, fill=white] { Token Stream };
        \node [block, left=2em of n2] (n3) { Error Handler };
        \draw [thick, <->, dashed] (n1) -- (n3);
        \node [block, right=2em of n2] (n4) { Symbol Table };
        \draw [thick, <->, dashed] (n1) -- (n4);
        \draw [thick, <->, dashed] (n2) -- (n3);
        \draw [thick, <->, dashed] (n2) -- (n4);
        \node [block, below=3em of n2] (n5) { Semantic Analysis };
        \draw [thick, ->] (n2) -- (n5) node [midway, fill=white] { Abstract Syntax Tree (AST) };
        \draw [thick, <->, dashed] (n3) -- (n5);
        \draw [thick, <->, dashed] (n4) -- (n5);
        \node [phantomblock, below=2em of n5] (n6) { \phantom{} };
        \draw [thick, ->] (n5) -- (n6) node [midway, fill=white] { Decorated AST };
    \end{tikzpicture}
    }
    \caption{Error handling and symbol table}
    \label{fig:error-handling-and-symbol-table}
\end{figure}

\subsection{Intermediate Code Generator}

To produce \textit{intermediate code} which allows:
\begin{itemize}
    \item \textit{Portability} (machine and language independence).
    \item \textit{Decoupling} between front-end and back-end.
    \item Allow \textit{transpiling} and \textit{optimization}.
\end{itemize}

\subsubsection{Intermediate Representation (IR)}

Three tiers of \textit{intermediate representation} (IR) in varying degree of abstraction:
\begin{itemize}
    \item High-level IR (HIR)
    \item Mid-level IR (MIR)
    \item Low-level IR (LIR)
\end{itemize}

Note that HIR is closest to AST and LIR is closest to machine code. Examples of LIR include:
\begin{itemize}
    \item Three Address Code (TAC)
    \item LLVM IR
    \item GCC Gimple
\end{itemize}

\subsection{Optimizer}

Rewrite \textit{intermediate code} to run faster and/or take up less space.

\subsection{Code Generation}

Output the program in the target language (usually assembly).

\subsection{Combined Phases}

\begin{figure}[H]
    \centering
    \fbox{
    \begin{tikzpicture}[auto]
        \node [dashedblock] (n0) { Source Code };
        \node [block, below=3em of n0] (n1) { Lexical Analysis (Lexer) };
        \node [block, below=3em of n1] (n2) { Syntax Analysis (Parser) };
        \draw [thick, ->] (n1) -- (n2) node [midway, fill=white] { Token Stream };
        \node [block, left=2em of n2] (n3) { Error Handler };
        \draw [thick, <->, dashed] (n1) -- (n3);
        \node [block, right=2em of n2] (n4) { Symbol Table };
        \draw [thick, <->, dashed] (n1) -- (n4);
        \draw [thick, <->, dashed] (n2) -- (n3);
        \draw [thick, <->, dashed] (n2) -- (n4);
        \node [block, below=3em of n2] (n5) { Semantic Analysis };
        \draw [thick, ->] (n2) -- (n5) node [midway, fill=white] { Abstract Syntax Tree (AST) };
        \draw [thick, <->, dashed] (n3) -- (n5);
        \draw [thick, <->, dashed] (n4) -- (n5);
        \node [block, below=3em of n5] (n6) { Optimizer };
        \draw [thick, ->] (n5) -- (n6) node [midway] { Decorated AST };
        \node [phantomblock, right=2em of n6] (n7) { IR };
        \draw [thick, ->] (n6) to [out=300,in=60] (n7);
        \draw [thick, ->] (n7) to [out=300,in=60] (n6);
        \node [block, below=3em of n6] (n8) { Code Generator };
        \draw [thick, ->] (n6) -- (n8) node [midway] { IR };
        \node [dashedblock, below=3em of n8] (n9) { Machine Code };
        \draw [thick, ->] (n8) -- (n9);
        \draw [thick, ->] (n0) -- (n1) node [midway] { Character Stream };
    \end{tikzpicture}
    }
    \caption{All phases of a compiler}
    \label{fig:compiler-all-phases}
\end{figure}

\subsection{Compiler versus Interpreter}

\begin{figure}[H]
    \centering
    \begin{subfigure}{0.45\textwidth}
        \centering
        \fbox{
        \begin{tikzpicture}
            \node [dashedblock] (n1) { Source Code };
            \node [block, below=2em of n1] (n2) { Executable Code };
            \node [block, below=2em of n2] (n3) { CPU };
            \draw [thick, ->] (n1) -- (n2) node [midway, right] 
                { preprocessing };
            \draw [thick, ->] (n2) -- (n3) node [midway, right]
                { execution };
        \end{tikzpicture}
        }
        \caption{Compiler}
        \label{fig:compiler-vs-interpreter:compiler}
    \end{subfigure}
    \begin{subfigure}{0.45\textwidth}
        \centering
        \fbox{
        \begin{tikzpicture}
            \node [dashedblock] (n1) { Source Code };
            \node [block, below=2em of n1] (n2) { IR Code };
            \node [block, below=2em of n2] (n3) { Interpreter };
            \node [block, below=2em of n3] (n4) { CPU };
            \draw [thick, ->] (n1) -- (n2) node [midway, right] 
                { preprocessing };
            \draw [thick, ->] (n2) -- (n3) node [midway, right]
                { execution };
            \draw [thick, ->] (n3) -- (n4) node [midway, right]
            { execution };
        \end{tikzpicture}
        }
        \caption{Interpreter}
        \label{fig:compiler-vs-interpreter:interpreter}
    \end{subfigure}
    \caption{Difference between a compiler and an interpreter}
    \label{fig:compiler-vs-interpreter}
\end{figure}
