\section{Lexical Analysis}

Extracts \textit{lexemes} from source code taken in as a stream of charcaters.

\begin{figure}[H]
    \centering
    \begin{tikzpicture}
        \node [phantomblock] (n1) { \phantom{} };
        \node [block, right=8em of n1] (n2) { Lexer };
        \node [phantomblock, right=8em of n2] (n3) { \phantom{} };
        \draw [thick, ->] (n1) -- (n2) node [above, near start]
            { Character stream };
        \draw [thick, ->] (n2) -- (n3) node [above, near end]
            { Token stream };
    \end{tikzpicture}
    \caption{Lexical Analysis I/O}
    \label{fig:lexical-analysis-io}
\end{figure}

\begin{definition}[Lexeme]
    A \textit{lexeme} is a sequence of characters which form a \textit{lexical unit} in the grammar of a given language.
\end{definition}

\begin{definition}[Delimiter]
    A \textit{delimiter} is a character or a sequence of characters which separates lexemes.
\end{definition}

\begin{remark}
    A \textit{lexeme} is analogous to a "word" within an English sentence separated by \textit{delimiters}, such as a "space".
\end{remark}

\subsection{Regular Expressions}

\textit{Regular expressions} (often abbreviated as \textit{regex}) can be used to describe the lexemes of a language.

To a recognize a \textit{regex}, one may use a \textit{Finite State Automata} (FSA), namely the \textit{Deterministic Finite Automata} (DFA), which is sufficiently powerful in recognizing \textit{regular} languages.

\subsubsection{Definition of a Language}

\begin{definition}[Alphabet]
    An \textit{alphabet} $\Sigma$ is a finite set of symbols.
\end{definition}

\begin{definition}[String]
    A \textit{string} over $\Sigma$ is a consecutive sequence of symbols from alphabet $\Sigma$.
\end{definition}

\begin{definition}[Language]
    A \textit{language} $L$ is a set of strings over an alphabet $\Sigma$.
\end{definition}

\subsubsection{Regular Operations}

\begin{definition}[Regular Operations]
    Given languages $L_0$ and $L_1$, there are three fundamental \textit{regular operations} defined as
    \begin{enumerate}
        \item \textbf{Union} ($L_0 \cup L_1$) (also denoted $L_0 \mid L_1$)
            \begin{equation}
                L_0 \cup L_1 \coloneqq 
                    \set{ x \given x \in L_0 \lor x \in L_1 }
            \end{equation}
        \item \textbf{Concatenation} ($L_0 \circ L_1$) (also denoted $L_0 L_1$)
            \begin{equation}
                L_0 \circ L_1 \coloneqq
                    \set{ xy \given x \in L_0 \land y \in L_1 }
            \end{equation}
        \item \textbf{Kleene Star} ($L_0^{\ast}$)
            \begin{equation}
                L_0^{\ast} \coloneqq
                    \set{ x_1 x_2 \cdots x_k 
                        \given k > 0 \land x_i \in L_0
                    } \cup \set{ \epsilon }
            \end{equation}
    \end{enumerate}
\end{definition}

\subsubsection{Regular Expression Formal Definition}

\begin{definition}[Regular Expression]
    For a given alphabet $\Sigma$, the \textit{regular expression} $R$ is defined inductively as
    \begin{enumerate}
        \item $a \in \Sigma$
        \item $\epsilon$
        \item $\emptyset$
        \item $(R_1 \mid R_2)$
        \item $(R_1 R_2)$
        \item $(R_1^{\ast})$
    \end{enumerate}
    Where
    \begin{itemize}
        \item $R_1$, $R_2$ are valid \textit{regular expressions}.
        \item $\epsilon$ denotes the empty string.
        \item $\emptyset$ is the empty set.
    \end{itemize}
\end{definition}

\begin{remark}
    Note that \textit{parantheses} are need for the inductive cases to prevent ambiguity. This definition assumes that the concatenation, union and Kleene star operators have equal precedence.
\end{remark}

\begin{example}
    Given alphabet $\Sigma = \set{a, b}$, then
    \begin{table}[H]    
        \centering
        \begin{tabular}{@{} r  l @{}}
            \toprule
            Regular Expression $R$ & $L(R)$ \\
            \midrule
            $a$         & $\set{a}$ \\
            $a \mid b$  & $\set{a, b}$ \\
            $(ab)^\ast$ & 
                $\set{\epsilon, ab, abab, ababab, \dots}$ \\
            $(a \mid \epsilon)b$ & $\set{b, ab, aab, \dots}$ \\
            $(a \mid b^\ast)a$ & $\set{a, aa, ba, bba, \dots}$ \\
            \bottomrule
        \end{tabular}
    \end{table}
    
    Note that $L(R)$ denotes "the \textit{language} $L$ of the \textit{regex} $R$".
\end{example}

\subsubsection{Associativity}

To reduce the number of \textit{parantheses} required to unambiguously interpret a \textit{regex}, it is possible to assign different \textit{priorities} to the three operators $\ast$, $\circ$ and $\mid$.

\begin{definition}[Regular Operator Precedence]
    For the regular operators,
    \begin{figure}[H]
        \centering
        \begin{tabular}{@{} r | l @{}}
            \toprule
            Operator Precedence & Operator              \\
            \midrule
            Highest $\uparrow$  & Kleene Star $\ast$    \\
            \phantom{}          & Concatenation $\circ$ \\
            Lowest $\downarrow$ & Union $\cup$          \\
            \bottomrule
        \end{tabular}
        \caption{Regular Operator Precedence}
        \label{tab:regex-operator-precedence}
    \end{figure}
\end{definition}

\begin{remark}
    \textit{Precedence} can be thought as "binding power", that is, higher precedence implies higher binding power, and vice versa.
\end{remark}

\begin{example}
    The regex 
    \begin{equation*}
        ab^\ast \mid cd
    \end{equation*}
    Is equivalently
    \begin{equation*}
        ((a \circ (b^\ast)) \mid (c \circ d))
    \end{equation*}
\end{example}
\subsubsection{Syntax Sugars}

\begin{definition}[Syntax Sugar]
    \textit{Syntax sugar} for \textit{regexes} are introduced as aliases to make the regular language easier to express. However, syntax sugars do not otherwise change the expressiveness of regular languages -- they are simply \textit{aliases}.
\end{definition}

\begin{example}
    Given the alphabet of the lowercase Roman alphabet $\Sigma$, one can introduce several syntax sugars operator and expressions
    \begin{figure}[H]
        \centering
        \begin{tabular}{@{} r | l @{}}
            \toprule
            Syntax Sugar & Equivalent Regular Expression \\
            \midrule
            $a^+$ & $aa^\ast$  \\
            $a?$ & $a \mid \epsilon$ \\
            $[abc]$ & $a \mid b \mid c$ \\
            $[a \mhyphen z]$ & 
                $a \mid b \mid c \mid \cdots \mid z \equiv \Sigma$ \\
            $.$ & 
                $a \mid b \mid c \mid \cdots \mid z \equiv \Sigma$ \\
            $[^{\wedge} a \mhyphen c]$ & 
                $d \mid \cdots \mid z \equiv \Sigma - \set{a, b, c}$ \\
            \bottomrule
        \end{tabular}
        \caption{Syntax sugar for the lowercase Roman alphabet}
        \label{fig:roman-syntax-sugar}
    \end{figure}
\end{example}

\begin{remark}
    Operators introduced means there is a need to escape the \textit{meta-characters}, the characters which themselves are used to represent the regular operators.
    
    For example, to use the meta-character $\mid$ as an input character for a regular expression, one could escape it as $\backslash\vert$.
\end{remark}

\subsubsection{Ambiguity}

\begin{definition}[Ambiguous]
    A regex is \textit{ambiguous} if it recognizes input string in multiple ways.
\end{definition}

\begin{example}
    The regex $(00)^\ast (000)^\ast$ is \textit{ambiguous} since it can recognize the input string \enquote{$000000$} as either
    \begin{itemize}
        \item $000 \circ 000$, or
        \item $00 \circ 00 \circ 00$.
    \end{itemize}
\end{example}

\begin{example}
    A more significant example would be in the case of keywords vs identifiers in a programming language. For example,
    \begin{figure}[H]
        \centering
        \begin{forest}
            [\texttt{ifa:=0}
                [\boxed{
                    \texttt{<if> <a> <:=> <0> <;>}
                }]
                [\boxed{
                    \texttt{<ifa> <:=> <0> <;>}
                }]
            ]
        \end{forest}
        \caption{Ambiguity in keywords vs identifiers}
    \end{figure}
\end{example}

It is then necessary to disambiguate between different possible matches and only retain a single matching pattern.

\begin{definition}[Disambiguation]
    Two rules are used for \textit{disambiguation}:
    \begin{enumerate}
        \item \textbf{Longest Match}
        \item \textbf{Rule Priority} (order)
    \end{enumerate}
\end{definition}

\subsubsection{Greedy Matching}

\begin{definition}[Greedy Matching]
    \textit{Greedy matching} refers to the default behaviour for regexes to recognize as much of the input as possible. For instance, to match HTML tags such as
    \begin{figure}[H]
        \centering
        \begin{cminted}{html}
        <a href="/"> Test </a>
        \end{cminted}
    \end{figure}
    
    The pattern
    \begin{equation*}
        \langle \Sigma^+ \rangle
    \end{equation*}
    
    Matches the entire input sequence
    \begin{equation*}
        \boxed{\texttt{<a href="/"> Test </a>}}
    \end{equation*}
    
    To match only the tags themselves, one would need to exclude the opening and closing angle brackets
    \begin{equation*}
        \langle (\Sigma - \set{\langle, \rangle})^+ \rangle
    \end{equation*}
    
    Which would yield the desired match
    \begin{equation*}
        \boxed{\texttt{<a href="/">}}
        \texttt{ Test }
        \boxed{\texttt{</a>}}
    \end{equation*}
\end{definition}

\subsubsection{Searching versus Lexing}

\begin{definition}[Searching]
    \textit{Searching} aims to find a substring which satisfies a regex and can ignore the rest of the input characters.
\end{definition}

\begin{remark}
    Note that while \textit{searching} aims to find \textit{a} substring which matches the supplied regex, \textit{lexing} aims to find \textit{all} substrings which matches the supplied regex.
\end{remark}

\subsection{Deterministic Finite Automata}

To recognize a \textit{regular expression}, one may use a \textit{Deterministic Finite Automata} (DFA).

\begin{definition}[Deterministic Finite Automata (DFA)]
    A \textit{Deterministic Finite Automata} (DFA) $M$ can be described via the 5-tuple
    \begin{equation}
        M = \langle 
            \Sigma, 
            \mathcal{Q}, 
            \delta, q_0, 
            \mathcal{F} 
        \rangle
    \end{equation}
    Where
    \begin{enumerate}
        \item \textbf{Alphabet}: $\Sigma$ is the finite set of characters in the \textit{alphabet}.
        \item \textbf{States}: $\mathcal{Q}$ is the finite set of \textit{states}.
        \item \textbf{Transition Function}: $\delta \colon \mathcal{Q} \times \Sigma \to \mathcal{Q}$. This function transitions from one \textit{state} to another on a suitable input character.
        \item \textbf{Start State}: $q_0 \in \mathcal{Q}$ is the \textit{start state}.
        \item \textbf{Final States}: $\mathcal{F} \subseteq \mathcal{Q}$ is the set of \textit{final states}.
    \end{enumerate}
\end{definition}

\subsubsection{DFA Notation}

A DFA can be represented graphically with the following notations.

\begin{definition}[DFA Notations]
    \begin{figure}[H]
        \centering
        \begin{subfigure}{0.24\textwidth}
            \centering
            \begin{tikzpicture}
                \node [state] (q_i) {$q_i$};
            \end{tikzpicture}
            \caption{State}
        \end{subfigure}
        \begin{subfigure}{0.24\textwidth}
            \centering
            \begin{tikzpicture}
                \node [state, initial] (q_0) {$q_0$};
            \end{tikzpicture}
            \caption{Start State}
        \end{subfigure}
        \begin{subfigure}{0.24\textwidth}
            \centering
            \begin{tikzpicture}
                \node [state, accepting] (q_f) {$q_f$};
            \end{tikzpicture}
            \caption{Final State}
        \end{subfigure}
        \begin{subfigure}{0.24\textwidth}
            \centering
            \begin{tikzpicture}
                \node [state] (q_1) {$q_1$};
                \node [state, right=2em of q_1] (q_2) {$q_2$};
                \path [->] (q_1) edge node [midway, above] {$x$} (q_2);
            \end{tikzpicture}
            \caption{Transition}
        \end{subfigure}
        \caption{DFA Notations}
        \label{fig:dfa-notations}
    \end{figure}
    
    Note that $q_1 \times a \to q_2$ denotes to transition from state $q_1$ to $q_2$ on reading input character $a$.
\end{definition}

\begin{remark}
    Note that DFA diagrams usually omit the trash state $q_{\textit{trash}}$, which is the implicit final state that rejects the read input string.
\end{remark}

\subsection{Non-deterministic Finite Automata}

A \textit{Non-deterministic Finite Automata} (NFA) is a DFA with added convience features that do not otherwise change its expressive power.

\begin{definition}[Non-deterministic Finite Automata (NFA)]
    A \textit{Non-deterministic Finite Automata} (NFA) $M$ can be described by the 5-tuple
    \begin{equation}
        M = \langle
            \Sigma,
            \mathcal{Q},
            \delta,
            q_0,
            \mathcal{F}
        \rangle
    \end{equation}
    
    Where
    \begin{enumerate}
        \item \textbf{Alphabet}: $\Sigma$ is the finite set of characters making up the \textit{alphabet}.
        \item \textbf{States}: $\mathcal{Q}$ is the finite set of \textit{states}.
        \item \textbf{Transition Function}: $\delta \colon \mathcal{Q} \times \Sigma_{\epsilon} \to \Pow{\mathcal{Q}}$.
        \item \textbf{Start State}: $q_0 \in \mathcal{Q}$.
        \item \textbf{Final States}: $\mathcal{F} \subseteq \mathcal{Q}$.
    \end{enumerate}
    
    Note that $\Sigma_{\epsilon} \coloneqq \Sigma \cup \set{\epsilon}$ and $\Pow{\cdot}$ is the power set.
\end{definition}

\begin{remark}
    Special attention needs to be paid to the \textit{transition function} of the NFA. Notice that for the transition function $\delta$
    \begin{equation}
        \delta \colon \mathcal{Q} \times \Sigma_{\epsilon} \to \Pow{\mathcal{Q}}
    \end{equation}

    \begin{itemize}
        \item There may be $\epsilon$-transitions (read as \textit{epsilon}-transitions). The NFA may change states \textit{without} reading input.
        \item On a single input character, the NFA can transition to \textit{multiple} states.
        \item On a single input character, the NFA can also go to \textit{no} state ($\emptyset$).
    \end{itemize}
    
    These convenience features combined gives rise to the \textit{non-determinism} in the name of the NFA.
    
    \begin{figure}[H]
        \centering
        \begin{subfigure}{0.45\textwidth}
            \centering
            \begin{tikzpicture}
            \node [state] (q_1) {$q_1$};
            \node [state, right=2em of q_1] (q_2) {$q_2$};
            \draw [->] (q_1) edge node [midway, above] {$\epsilon$} (q_2);
            \end{tikzpicture}
            \caption{Epsilon Transition}
        \end{subfigure}
        \begin{subfigure}{0.45\textwidth}
            \centering
            \begin{tikzpicture}
            \node [state] (q_1) {$q_1$};
            \node [state, above right=2em of q_1] (q_2) {$q_2$};
            \node [state, below right=2em of q_1] (q_3) {$q_2$};
            \draw [->] (q_1) edge node [midway, above] {$x$} (q_2);
            \draw [->] (q_1) edge node [midway, above] {$x$} (q_3);
            \end{tikzpicture}
            \caption{Transition to Multiple States}
        \end{subfigure}
        \caption{NFA Convenience Features}
        \label{fig:nfa-notation}
    \end{figure}
\end{remark}

\begin{definition}[NFA Acceptance]
    An NFA \textit{accepts} an input string if it is able to reach a final state upon consuming the input string.
\end{definition}

\subsection{DFA versus NFA}

\begin{figure}[H]
    \centering
    \begin{tabular}{@{} p{10em} p{10em} p{10em} @{}}
        \toprule
        Aspect & DFA & NFA \\
        \midrule
        For each input character 
        & Definitive next state 
        & Can either take $\epsilon$-transition or take which one of multiple possible transition. \\
        Acceptance
        & Ends in final state
        & Has a path from start state to final state \\
        Implementation 
        & Table-driven
        & Set of possible states \\
        Execution
        & Faster and less space needed
        & Slower and more space needed \\
        \bottomrule
    \end{tabular}
    \caption{DFA versus NFA}
    \label{fig:dfa-vs-nfa}
\end{figure}

\begin{remark}
    Note that NFA often has fewer nodes and edges compared to a DFA due to its non-determinism.
\end{remark}

\subsection{Implementing Lexing Automata}

\subsubsection{Overview}

Implementing a lexer specification involves the following process.

\begin{figure}[H]
    \centering
    \fbox{
    \begin{tikzpicture}
        \node [block] (lex_spec) { Lexical Specification };
        \node [block, below=2em of lex_spec] (regex) 
            { Regular Expression (Regex) };
        \node [block, below=2em of regex] (NFA) 
            { Non-deterministic Finite Automata (NFA) };
        \node [block, below=2em of NFA] (DFA)
            { Deterministic Finite Automata (DFA) };
        \node [block, below=2em of DFA] (TDI)
            { Table-driven DFA Implementation };
        \draw [->] (lex_spec) -- (regex);
        \draw [->] (regex) -- (NFA);
        \draw [->] (NFA) -- (DFA);
        \draw [->] (DFA) -- (TDI);
    \end{tikzpicture}
    }
    \caption{Lexing Automata Implementation}
    \label{fig:lexing-automata-overview}
\end{figure}

\subsection{Regular Expression to NFA Conversion (Thompson's Construction)}

The algorithm for converting regex to NFA is called  \textit{Thompson's Construction}.

\begin{definition}[Thompson's Construction]
    Given regex $A$, let $N(A)$ denote the application of the rules of \textit{Thompson's Construction}.
    
    Let $A$ have a single start and final state.
    \begin{figure}[H]
        \centering
        \begin{tikzpicture}
            \node [state, initial] (q_s) {};
            \node [phantomblock, right=2em of q_s] (q_p) 
                { \phantom{} };
            \node [phantomblock, above=0.5em of q_p] (q_a) 
                {$A$};
            \node [state, accepting, right=2em of q_p] (q_f) {};
            \draw [->] (q_s) -- (q_p);
            \draw [->] (q_p) -- (q_f);
            \node [draw=black, fit=(q_s) (q_f) (q_p) (q_a)] {};
        \end{tikzpicture}
        \caption{Regex $A$}
        \label{fig:thompson-notation}
    \end{figure}
    
    \textbf{CASE 1}: Single Symbol.
    
    Transition for a single input character $a \in \Sigma$ (note that $\epsilon$ is simply a symbol).
    
    \begin{figure}[H]
        \centering
        \begin{tikzpicture}
            \node [state, initial] (q_s) {};
            \node [state, accepting, right=2em of q_s] (q_f) {};
            \draw [->] (q_s) edge node [midway, above] { $a$ } (q_f);
        \end{tikzpicture}
        \caption{Single Symbol Case}
        \label{fig:thompson-single-symbol}
    \end{figure}
    
    \textbf{CASE 2}: Concatenation.
    
    Given two regexes $A$ and $B$, then $AB$ is simply $A$'s final state \textit{definalized}, and merged as $B$'s start state. That is, $A$'s final acceptance state is no longer accepting and becomes $B$'s start state.
    
    \begin{figure}[H]
        \centering
        \begin{tikzpicture}
            \node [state, initial] (qa_s) {};
            \node [phantomblock, right=2em of qa_s] (qa_p) 
                { \phantom{} };
            \node [phantomblock, above=0.5em of qa_p] (qa_a) 
                {$N(A)$};
            \node [state, dashed, right=2em of qa_p] (qa_f) {};
            \draw [->] (qa_s) -- (qa_p);
            \draw [->] (qa_p) -- (qa_f);
            \node [draw=black, fit=(qa_s) (qa_f) (qa_p) (qa_a)] {};
            \node [state, dashed, right=3em of qa_f] (qb_s) {};
            \node [phantomblock, right=2em of qb_s] (qb_p) 
                { \phantom{} };
            \node [phantomblock, above=0.5em of qb_p] (qb_a) 
                {$N(B)$};
            \node [state, accepting, right=2em of qb_p] (qb_f) {};
            \draw [->] (qb_s) -- (qb_p);
            \draw [->] (qb_p) -- (qb_f);
            \node [draw=black, fit=(qb_s) (qb_f) (qb_p) (qb_a)] {};
            \draw [-, dashed] (qa_f) -- (qb_s) node [midway, above] {$\equiv$};
        \end{tikzpicture}
        \caption{Concatenation Case}
        \label{fig:thompson-concat}
    \end{figure}
    
    \textbf{CASE 3}: Union (Alternation).
    
    Given two regexes $A$, $B$, then the union $A \mid B$ can be converted to NFA by:
    \begin{enumerate}
        \item Introduce new start state.
        \item Definalize $A$ and $B$'s final state.
        \item Introduce new final state.
        \item Connect the new start state, new final state, and $A$ and $B$ with $\epsilon$-transitions.
    \end{enumerate}
    
    \begin{figure}[H]
        \centering
        \begin{tikzpicture}[node distance=6em]
            \node [state, initial]   (q_s) {};
            \node [state]            (qa_s) [above right of=q_s] {};
            \node [state]            (qb_s) [below right of=q_s] {};
            \node [phantomblock]     (qa_p) [right of=qa_s] {};
            \node [phantomblock]     (qa_l) [above of=qa_p, node distance=1em] { $N(A)$ };
            \node [phantomblock]     (qb_p) [right of=qb_s] {};
            \node [phantomblock]     (qb_l) [above of=qb_p, node distance=1em] { $N(B)$ };
            \node [state]            (qa_f) [right of=qa_p] {};
            \node [state]            (qb_f) [right of=qb_p] {};
            \node [state, accepting] (q_f)  [below right of=qa_f] {};
            \draw [->] (q_s) edge [above] node[midway] {$\epsilon$} (qa_s);
            \draw [->] (q_s) edge [above] node[midway] {$\epsilon$} (qb_s);
            \draw [->] (qa_s) -- (qa_p);
            \draw [->] (qa_p) -- (qa_f);
            \draw [->] (qb_s) -- (qb_p);
            \draw [->] (qb_p) -- (qb_f);
            \draw [->] (qa_f) edge [above] node[midway] {$\epsilon$} (q_f);
            \draw [->] (qb_f) edge [above] node[midway] {$\epsilon$} (q_f);
            \node [draw=black, fit=(qa_s) (qa_p) (qa_l) (qa_f)] {};
            \node [draw=black, fit=(qb_s) (qb_p) (qb_l) (qb_f)] {};
        \end{tikzpicture}
        \caption{Union Case}
        \label{fig:thompson-union}
    \end{figure}
    
    \textbf{CASE 4}: Kleene Star.
    
    Given regex $A$, then its Kleene star $A^\ast$ can be converted to NFA by connecting $N(A)$'s start state and final state.
    
    \begin{figure}[H]
        \centering
        \begin{tikzpicture}
            \node [state, initial, initial text=] (q_s) {};
            \node [phantomblock, right=2em of q_s] (q_p) {};
            \node [state, accepting, right=2em of q_p] (q_f) {};
            \node [phantomblock, above=0.1em of q_p] (q_l) {$N(A)$};
            \draw [->] (q_s) -- (q_p);
            \draw [->] (q_p) -- (q_f);
            \path [->] (q_s) edge [bend left=60, above] 
                node [midway] {$\epsilon$} (q_f);
            \path [->] (q_f) edge [bend left=60, below] 
                node [midway] {$\epsilon$} (q_s);
            \node [draw=black, fit=(q_s) (q_f), inner sep=3em] {};
        \end{tikzpicture}
        \caption{Kleene Star Case}
        \label{fig:thompson-kleene-star}
    \end{figure}
\end{definition}

\subsection{NFA to DFA Conversion}

It is also possible to convert from NFA to DFA as they have equivalent power.

\begin{definition}[NFA to DFA Conversion]
    Given NFA $\langle \Sigma, \mathcal{Q}_N, \delta_N, q_N, \mathcal{F}_N \rangle$, it is possible to construct an equivalent DFA $\langle \Sigma, \mathcal{Q}_D, \delta_D, q_D, \mathcal{F}_D \rangle$.
    
    Each state $q_D$ of the DFA is a subset of the states of the NFA.
    \begin{equation}
        \Forall q_D \in \mathcal{Q}_D \colon q_D \subseteq \mathcal{Q}_N
    \end{equation}
\end{definition}

\begin{definition}[Epsilon Closure]
    The \textit{$\epsilon$-closure} of a state $q$, denoted $\epsilon(q)$, is the set of NFA states which are reachable from $q$ with $\epsilon$-transitions.
    
    \begin{equation}
        \epsilon(q) = \left\lbrace r \in \mathcal{Q} \mid q \xrightarrow[\epsilon]{\ast} r \right\rbrace
    \end{equation}
    
    Note that $\epsilon(q) \ne \emptyset$ since the state $q$ is itself in its $\epsilon$ closure, that is $q \in \epsilon(q)$.
\end{definition}

\begin{definition}[NFA to DFA High Level Algorithm]
    The target is to convert from
    \begin{equation}
        NFA = \langle
            \Sigma,
            \mathcal{Q}_N,
            \delta_N,
            s_N,
            \mathcal{F}_N
        \rangle \implies
        \langle
            \Sigma,
            \mathcal{Q}_D,
            \delta_D,
            s_D,
            \mathcal{F}_D
        \rangle = DFA
    \end{equation}
    
    Where
    \begin{equation}
        s_D \equiv \epsilon(s_N)
    \end{equation}
    
    For each DFA state $q_D \in \mathcal{Q}_D$, add the transition
    \begin{equation}
        q_D \times a \to T \subseteq \mathcal{Q}_N
    \end{equation}
    To $\delta_D$ if there exists a state $s$ where $T$ is
    \begin{equation}
        T = \epsilon(\delta_N (s, a))
    \end{equation}
    
    If some $q_D \in \mathcal{Q}_D$ contains one of the NFA's final states in $\mathcal{F}_N$, then mark $q_D$ as final as well, that is $q_D \in \mathcal{F}_D$.
\end{definition}

\begin{definition}[NFA to DFA: addTransition]
    To compute the \textit{transitions} between the DFA states, the procedure \texttt{addTransition} can be used on
    \begin{equation}
        q_D \in \mathcal{Q}_D \subseteq \Pow{\mathcal{Q}_N}
    \end{equation}
    
    Then \texttt{addTransition} has the signature
    \begin{equation}
        \mathtt{addTransition} \colon \mathcal{Q}_D \times \Sigma \to \mathcal{Q}_D \times \delta_D
    \end{equation}
    
    \begin{algorithm}[H]
        \begin{algorithmic}[1]
            \Procedure{addTransition}{$\langle q_D$, $a \rangle$}
                \State $q_D^\prime \gets \emptyset$
                \ForAll{$q_N \in q_D$}
                    \State $q_D^\prime \gets q_D^\prime \cup \epsilon(\delta_N(q_N, a))$
                \EndFor
                \State $\delta_D \gets \set{q_D \times a \to q_D^\prime}$
                \State \Return $\langle q_D^\prime, \delta_D \rangle$
            \EndProcedure
        \end{algorithmic}
        \caption{Compute transitions $\delta_D$ for a DFA state $q_D$ given input $a$}
        \label{algo:nfa-dfa-add-transition}
    \end{algorithm}
\end{definition}

\begin{definition}[NFA to DFA: Conversion Algorithm]
    Then to convert from a NFA to DFA:
    \begin{algorithm}[H]
        \begin{algorithmic}[1]
            \Procedure{convertNfaToDfa}{$
                \langle
                    \Sigma,
                    \mathcal{Q}_N,
                    \delta_N,
                    s_N,
                    \mathcal{F}_N
                \rangle
            $}
                \State $s_D \gets \epsilon(s_N)$
                \State $\delta_D \gets \emptyset$
                \State $\mathcal{Q}_D \gets \set{s_D}$
                \State \Call{WC.enqueue}{$s_D$} \Comment{$\mathrm{WC}$ is the workset.}
                \While{$\neg$ \Call{WC.isEmpty}{\null}}
                    \State $q_D \gets \Call{WC.dequeue}{\null}$
                    \ForAll{$a \in \Sigma$}
                        \State $\langle q_D^\prime, \delta_D^\prime \rangle \gets \Call{addTransition}{q_D, a}$
                        \State $\delta_D \gets \delta_D \cup \set{\delta_D^\prime}$
                        \If{$q_d^\prime \not\in \mathcal{Q}_D$}
                            \State \Call{WC.enqueue}{$q_D^\prime$}
                        \EndIf
                        \State $\mathcal{Q}_D \gets \mathcal{Q}_D \cup \set{q_D^\prime}$
                    \EndFor
                \EndWhile
                \State $\mathcal{F}_D \gets \set{x \in \mathcal{Q}_D \given x \cap \mathcal{F}_N \ne \emptyset}$
                \State \Return $\langle
                    \Sigma,
                    \mathcal{Q}_D,
                    \delta_D,
                    s_D,
                    \mathcal{F}_D
                \rangle$
            \EndProcedure
        \end{algorithmic}
        \caption{Convert NFA to DFA.}
        \label{algo:nfa-to-dfa}
    \end{algorithm}
\end{definition}

% TODO: Add NFA -> DFA conversion example

\subsubsection{DFA Implementation}

\begin{definition}[Table-driven DFA]
    A DFA can be implemented with a 2D table $T[\mathcal{Q}, \Sigma]$. The two dimensions are for
    \begin{enumerate}
        \item Set of states $\mathcal{Q}$.
        \item Alphabet $\Sigma$.
    \end{enumerate}
    
    Transitions of the form $s_i \xrightarrow{a} s_j$ corresponds to the table entry
    \begin{equation}
        T[i, a] = j
    \end{equation}
    
    When the DFA is in state $s_i$, upon reading input character $a$, it is then trivial to find the next state $s_j = T[i, a]$.
\end{definition}

% TODO: Add DFA -> minified DFA

\subsection{Finite State Transducers}

\begin{definition}
    The \textit{Finite State Transducer} (FST) $M$ can be described by the 6-tuple
    \begin{equation}
        M = \langle
            \Sigma,
            \Gamma,
            \mathcal{Q},
            \delta,
            q_0,
            \mathcal{F}
        \rangle
    \end{equation}
    
    Where
    \begin{enumerate}
        \item \textbf{Input Alphabet}: $\Sigma$ is the set of characters making up the \textit{input alphabet}.
        \item \textbf{Output Alphabet}: $\Gamma$ is the set of characters making up the \textit{output alphabet}.
        \item \textbf{States}: $\mathcal{Q}$ is the finite set of \textit{states}.
        \item \textbf{Transition Function}: $\delta \colon \mathcal{Q} \times \Sigma_{\epsilon} \to \Gamma_{\epsilon} \times \mathcal{Q}$.
        \item \textbf{Start State}: $q_0 \in \mathcal{Q}$.
        \item \textbf{Final States}: $\mathcal{F} \subseteq \mathcal{Q}$.
    \end{enumerate}
    
    Note that $\Gamma_{\epsilon} \coloneqq \Gamma \cup \set{\epsilon}$ and $\Sigma_{\epsilon} \coloneqq \Sigma \cup \set{\epsilon}$.
\end{definition}

\subsubsection{FST Output Function}

There are two output functions defined implicitly in the previous definition of a FST.

\begin{definition}[Mealy Output Function]
    The \textit{Mealy output function} is implicit in the FSA's transition function, and is defined as
    \begin{equation}
        \mathcal{Q} \times \Sigma_{\epsilon} \to \Gamma_{\epsilon}
    \end{equation}
    
    This output function outputs on the \textit{edges}. Its edges are labeled as \enquote{$i/o$}, where $i \in \Sigma_{\epsilon}$ and $j \in \Gamma_{\epsilon}$.
    
    \begin{figure}[H]
        \centering
        \begin{tikzpicture}
            \node [state] (q_i) {$q_i$};
            \node [state, right=2em of q_i] (q_j) {$q_j$};
            \draw [->] (q_i) edge node [midway, above] {$i/o$} (q_j);
        \end{tikzpicture}
        \caption{Mealy Output Function}
        \label{fig:fst-mealy}
    \end{figure}
\end{definition}

\begin{definition}[Moore Output Function]
    The \textit{Moore output function} is also implicit in the FSA's transition function, and is defined as
    \begin{equation}
        \mathcal{Q} \to \Gamma_{\epsilon}
    \end{equation}
    
    This output function outputs upon entry to a state. Its nodes are labelled with output \enquote{$o$}.
    
    \begin{figure}[H]
        \centering
        \begin{tikzpicture}
            \node [state] (q_i) {$q_i/o$};
        \end{tikzpicture}
        \caption{Moore Output Function}
        \label{fig:fst-moore}
    \end{figure}
\end{definition}

\begin{remark}
    Note that FSTs allow us to label transitions into error states for error reporting.
\end{remark}
