\section{Lexical Analysis}

Extracts \textit{lexemes} from source code taken in as a stream of charcaters.

\begin{figure}[H]
    \centering
    \begin{tikzpicture}
        \node [phantomblock] (n1) { \phantom{} };
        \node [block, right=8em of n1] (n2) { Lexer };
        \node [phantomblock, right=8em of n2] (n3) { \phantom{} };
        \draw [thick, ->] (n1) -- (n2) node [above, near start]
            { Character stream };
        \draw [thick, ->] (n2) -- (n3) node [above, near end]
            { Token stream };
    \end{tikzpicture}
    \caption{Lexical Analysis I/O}
    \label{fig:lexical-analysis-io}
\end{figure}

\begin{definition}[Lexeme]
    A \textit{lexeme} is a sequence of characters which form a \textit{lexical unit} in the grammar of a given language.
\end{definition}

\begin{definition}[Delimiter]
    A \textit{delimiter} is a character or a sequence of characters which separates lexemes.
\end{definition}

\begin{remark}
    A \textit{lexeme} is analogous to a "word" within an English sentence separated by \textit{delimiters}, such as a "space".
\end{remark}

\subsection{Regular Expressions}

\textit{Regular expressions} (often abbreviated as \textit{regex}) can be used to describe the lexemes of a language.

To a recognize a \textit{regex}, one may use a \textit{Finite State Automata} (FSA), namely the \textit{Deterministic Finite Automata} (DFA), which is sufficiently powerful in recognizing \textit{regular} languages.

\subsubsection{Definition of a Language}

\begin{definition}[Alphabet]
    An \textit{alphabet} $\Sigma$ is a finite set of symbols.
\end{definition}

\begin{definition}[String]
    A \textit{string} over $\Sigma$ is a consecutive sequence of symbols from alphabet $\Sigma$.
\end{definition}

\begin{definition}[Language]
    A \textit{language} $L$ is a set of strings over an alphabet $\Sigma$.
\end{definition}

\subsubsection{Regular Operations}

\begin{definition}[Regular Operations]
    Given languages $L_0$ and $L_1$, there are three fundamental \textit{regular operations} defined as
    \begin{enumerate}
        \item \textbf{Union} ($L_0 \cup L_1$) (also denoted $L_0 \mid L_1$)
            \begin{equation}
                L_0 \cup L_1 \coloneqq 
                    \set{ x \given x \in L_0 \lor x \in L_1 }
            \end{equation}
        \item \textbf{Concatenation} ($L_0 \circ L_1$) (also denoted $L_0 L_1$)
            \begin{equation}
                L_0 \circ L_1 \coloneqq
                    \set{ xy \given x \in L_0 \land y \in L_1 }
            \end{equation}
        \item \textbf{Kleene Star} ($L_0^{\ast}$)
            \begin{equation}
                L_0^{\ast} \coloneqq
                    \set{ x_1 x_2 \cdots x_k 
                        \given k > 0 \land x_i \in L_0
                    } \cup \set{ \epsilon }
            \end{equation}
    \end{enumerate}
\end{definition}

\subsubsection{Regular Expression Formal Definition}

\begin{definition}[Regular Expression]
    For a given alphabet $\Sigma$, the \textit{regular expression} $R$ is defined inductively as
    \begin{enumerate}
        \item $a \in \Sigma$
        \item $\epsilon$
        \item $\emptyset$
        \item $(R_1 \mid R_2)$
        \item $(R_1 R_2)$
        \item $(R_1^{\ast})$
    \end{enumerate}
    Where
    \begin{itemize}
        \item $R_1$, $R_2$ are valid \textit{regular expressions}.
        \item $\epsilon$ denotes the empty string.
        \item $\emptyset$ is the empty set.
    \end{itemize}
\end{definition}

\begin{remark}
    Note that \textit{parantheses} are need for the inductive cases to prevent ambiguity. This definition assumes that the concatenation, union and Kleene star operators have equal precedence.
\end{remark}

\begin{example}
    Given alphabet $\Sigma = \set{a, b}$, then
    \begin{table}[H]
        \centering
        \begin{tabular}{@{} r  l @{}}
            \toprule
            Regular Expression $R$ & $L(R)$ \\
            \midrule
            $a$         & $\set{a}$ \\
            $a \mid b$  & $\set{a, b}$ \\
            $(ab)^\ast$ & 
                $\set{\epsilon, ab, abab, ababab, \dots}$ \\
            $(a \mid \epsilon)b$ & $\set{b, ab, aab, \dots}$ \\
            $(a \mid b^\ast)a$ & $\set{a, aa, ba, bba, \dots}$ \\
            \bottomrule
        \end{tabular}
    \end{table}
    
    Note that $L(R)$ denotes "the \textit{language} $L$ of the \textit{regex} $R$".
\end{example}

\subsubsection{Associativity}

To reduce the number of \textit{parantheses} required to unambiguously interpret a \textit{regex}, it is possible to assign different \textit{priorities} to the three operators $\ast$, $\circ$ and $\mid$.

\begin{definition}[Regular Operator Precedence]
    For the regular operators,
    \begin{figure}[H]
        \centering
        \begin{tabular}{@{} r | l @{}}
            \toprule
            Operator Precedence & Operator              \\
            \midrule
            Highest $\uparrow$  & Kleene Star $\ast$    \\
            \phantom{}          & Concatenation $\circ$ \\
            Lowest $\downarrow$ & Union $\cup$          \\
            \bottomrule
        \end{tabular}
        \caption{Regular Operator Precedence}
        \label{tab:regex-operator-precedence}
    \end{figure}
\end{definition}

\begin{remark}
    \textit{Precedence} can be thought as "binding power", that is, higher precedence implies higher binding power, and vice versa.
\end{remark}

\begin{example}
    The regex 
    \begin{equation*}
        ab^\ast \mid cd
    \end{equation*}
    Is equivalently
    \begin{equation*}
        ((a \circ (b^\ast)) \mid (c \circ d))
    \end{equation*}
\end{example}
\subsubsection{Syntax Sugars}

\begin{definition}[Syntax Sugar]
    \textit{Syntax sugar} for \textit{regexes} are introduced as aliases to make the regular language easier to express. However, syntax sugars do not otherwise change the expressiveness of regular languages -- they are simply \textit{aliases}.
\end{definition}

\begin{example}
    Given the alphabet of the lowercase Roman alphabet $\Sigma$, one can introduce several syntax sugars operator and expressions
    \begin{figure}[H]
        \centering
        \begin{tabular}{@{} r | l @{}}
            \toprule
            Syntax Sugar & Equivalent Regular Expression \\
            \midrule
            $a^+$ & $aa^\ast$  \\
            $a?$ & $a \mid \epsilon$ \\
            $[abc]$ & $a \mid b \mid c$ \\
            $[a \mhyphen z]$ & 
                $a \mid b \mid c \mid \cdots \mid z \equiv \Sigma$ \\
            $.$ & 
                $a \mid b \mid c \mid \cdots \mid z \equiv \Sigma$ \\
            $[^{\wedge} a \mhyphen c]$ & 
                $d \mid \cdots \mid z \equiv \Sigma - \set{a, b, c}$ \\
            \bottomrule
        \end{tabular}
        \caption{Syntax sugar for the lowercase Roman alphabet}
        \label{fig:roman-syntax-sugar}
    \end{figure}
\end{example}

\begin{remark}
    Operators introduced means there is a need to escape the \textit{meta-characters}, the characters which themselves are used to represent the regular operators.
    
    For example, to use the meta-character $\mid$ as an input character for a regular expression, one could escape it as $\backslash\vert$.
\end{remark}

\subsubsection{Ambiguity}

\begin{definition}[Ambiguous]
    A regex is \textit{ambiguous} if it recognizes input string in multiple ways.
\end{definition}

\begin{example}
    The regex $(00)^\ast (000)^\ast$ is \textit{ambiguous} since it can recognize the input string \enquote{$000000$} as either
    \begin{itemize}
        \item $000 \circ 000$, or
        \item $00 \circ 00 \circ 00$.
    \end{itemize}
\end{example}

\begin{example}
    A more significant example would be in the case of keywords vs identifiers in a programming language. For example,
    \begin{figure}[H]
        \centering
        \begin{forest}
            [\texttt{ifa:=0}
                [\boxed{
                    \texttt{<if> <a> <:=> <0> <;>}
                }]
                [\boxed{
                    \texttt{<ifa> <:=> <0> <;>}
                }]
            ]
        \end{forest}
        \caption{Ambiguity in keywords vs identifiers}
    \end{figure}
\end{example}

It is then necessary to disambiguate between different possible matches and only retain a single matching pattern.

\begin{definition}[Disambiguation]
    Two rules are used for \textit{disambiguation}:
    \begin{enumerate}
        \item \textbf{Longest Match}
        \item \textbf{Rule Priority} (order)
    \end{enumerate}
\end{definition}

\subsubsection{Greedy Matching}

\begin{definition}[Greedy Matching]
    \textit{Greedy matching} refers to the default behaviour for regexes to recognize as much of the input as possible. For instance, to match HTML tags such as
    \begin{figure}[H]
        \centering
        \begin{cminted}{html}
        <a href="/"> Test </a>
        \end{cminted}
    \end{figure}
    
    The pattern
    \begin{equation*}
        \langle \Sigma^+ \rangle
    \end{equation*}
    
    Matches the entire input sequence
    \begin{equation*}
        \boxed{\texttt{<a href="/"> Test </a>}}
    \end{equation*}
    
    To match only the tags themselves, one would need to exclude the opening and closing angle brackets
    \begin{equation*}
        \langle (\Sigma - \set{\langle, \rangle})^+ \rangle
    \end{equation*}
    
    Which would yield the desired match
    \begin{equation*}
        \boxed{\texttt{<a href="/">}}
        \texttt{ Test }
        \boxed{\texttt{</a>}}
    \end{equation*}
\end{definition}

\subsubsection{Searching versus Lexing}

\begin{definition}[Searching]
    \textit{Searching} aims to find a substring which satisfies a regex and can ignore the rest of the input characters.
\end{definition}

\begin{remark}
    Note that while \textit{searching} aims to find \textit{a} substring which matches the supplied regex, \textit{lexing} aims to find \textit{all} substrings which matches the supplied regex.
\end{remark}

\subsection{Deterministic Finite Automata}

To recognize a \textit{regular expression}, one may use a \textit{Deterministic Finite Automata} (DFA).

\begin{definition}[Deterministic Finite Automata (DFA)]
    A \textit{Deterministic Finite Automata} (DFA) $M$ can be described via the 5-tuple
    \begin{equation}
        M = \langle 
            \Sigma, 
            \mathcal{Q}, 
            \delta, q_0, 
            \mathcal{F} 
        \rangle
    \end{equation}
    Where
    \begin{enumerate}
        \item \textbf{Alphabet}: $\Sigma$ is the finite set of characters in the \textit{alphabet}.
        \item \textbf{States}: $\mathcal{Q}$ is the finite set of \textit{states}.
        \item \textbf{Transition Function}: $\delta \colon \mathcal{Q} \times \Sigma \to \mathcal{Q}$. This function transitions from one \textit{state} to another on a suitable input character.
        \item \textbf{Start State}: $q_0 \in \mathcal{Q}$ is the \textit{start state}.
        \item \textbf{Final States}: $\mathcal{F} \subseteq \mathcal{Q}$ is the set of \textit{final states}.
    \end{enumerate}
\end{definition}

\subsubsection{DFA Notation}

A DFA can be represented graphically with the following notations.

\begin{definition}[DFA Notations]
    \begin{figure}[H]
        \centering
        \begin{subfigure}{0.24\textwidth}
            \centering
            \begin{tikzpicture}
                \node [state] (q_i) {$q_i$};
            \end{tikzpicture}
            \caption{State}
        \end{subfigure}
        \begin{subfigure}{0.24\textwidth}
            \centering
            \begin{tikzpicture}
                \node [state, initial] (q_0) {$q_0$};
            \end{tikzpicture}
            \caption{Start State}
        \end{subfigure}
        \begin{subfigure}{0.24\textwidth}
            \centering
            \begin{tikzpicture}
                \node [state, accepting] (q_f) {$q_f$};
            \end{tikzpicture}
            \caption{Final State}
        \end{subfigure}
        \begin{subfigure}{0.24\textwidth}
            \centering
            \begin{tikzpicture}
                \node [state] (q_1) {$q_1$};
                \node [state, right=2em of q_1] (q_2) {$q_2$};
                \path [->] (q_1) edge node [midway, above] {$x$} (q_2);
            \end{tikzpicture}
            \caption{Transition}
        \end{subfigure}
        \caption{DFA Notations}
        \label{fig:dfa-notations}
    \end{figure}
    
    Note that $q_1 \times a \to q_2$ denotes to transition from state $q_1$ to $q_2$ on reading input character $a$.
\end{definition}

\begin{remark}
    Note that DFA diagrams usually omit the trash state $q_{\textit{trash}}$, which is the implicit final state that rejects the read input string.
\end{remark}
