\section{Optimization}

\subsection{Basics}

\begin{definition}[Ideal Characteristics of Target Code]
    The \textit{ideal target code} has the characteristics
    \begin{itemize}
        \item Semantics-preserving
        \item Efficient
        \item Small memory footprint
        \item Small space usage
    \end{itemize}
\end{definition}

\begin{definition}[Semantic-Preserving]
    \textit{Optimization} as an \textit{improvement} requires that the semantics of the source code is strictly \textit{preserved}; only space and/or time requirements are improved in the target code.
\end{definition}

\begin{definition}[Side-Effects]
    A function $f$ has \textit{side-effects} if it
    \begin{itemize}
        \item Modifies external state, or
        \item Interacts with calling functions or the outside world, or
        \item Produces different output given the same input.
    \end{itemize}

    If a function $f$ does \textit{not} have \textit{side-effects}, then $f$ is considered a \textit{pure} function.
\end{definition}

\begin{remark}
    Side-effects need to be carefully considered when trying to optimize function calls.
    
    For instance, it is only safe to rewrite
    \begin{minted}{text}
f(x) + f(x)
    \end{minted}
    
    With
    \begin{minted}{text}
2 * f(x)
    \end{minted}
    
    If and only if $f$ is a \textit{pure} function.
\end{remark}

\begin{definition}[Optimization at Different Levels]
    Different information is available for optimization at different levels
    \begin{itemize}
        \item High level
        \begin{itemize}
            \item Type information
        \end{itemize}
        \item Intermediate level
        \begin{itemize}
            \item Control Flow Analysis
            \item Data Flow Analysis
            \item Index calculation
            \item Bounds checking
        \end{itemize}
        \item Low level
        \begin{itemize}
            \item Target code selection
            \item Register allocation
        \end{itemize}
    \end{itemize}
\end{definition}

\begin{remark}
    There is a classic conflict in optimization: the trade-off between \textit{time} and \textit{space}. But it is generally the case that the fewer instructions there are, the faster the execution.
\end{remark}

\begin{definition}[Common Peephole Optimizations]
    There are numerous optimizations which are common at the source, AST or IR level:
    \begin{itemize}
        \item Common Subexpression Elimination
        \item Algebraic Identities
        \item Dead Code Elimination
        \item Constant Folding
        \item Strength Reduction
        \item Inlining
        \item Copy Propagation
        \item Loop optimizations
        \begin{itemize}
            \item Unrolling
            \item Fusion / Fission
            \item Code Motion
            \item Tiling
            \item Inversion
            \item Interchange
            \item Unswitching
        \end{itemize}
    \end{itemize}
\end{definition}

\subsection{Common Subexpression Elimination}

\begin{definition}[Common Subexpression]
    An expression $E$ is termed a \textit{common subexpression} if it has been previously computed, and the variables involved in the subexpression has not changed since the computation.
\end{definition}

\begin{definition}[Common Subexpression Elimination]
    Introduce a new temporary, storing the value computed by the common subexpression.
\end{definition}

\begin{example}
    For instance, given the code fragment
    \begin{minted}{java}
int a = x + 42 * d;
int b = y + 42 * d;
    \end{minted}
    
    The common subexpression \texttt{42 * d} can be factored out to give
    \begin{minted}{java}
int t1 = 42 * d;
int a = x + t1;
int b = y + t1;
    \end{minted}
\end{example}

\subsection{Algebraic Identities}

\begin{definition}[Algebraic Identities Optimization]
    Simple algebraic identities such as
    \begin{align}
        x \times 0 &= 0 \\
        x \times 1 &= 1
    \end{align}
    
    Can be used to simplify trivial expressions such as
    \begin{itemize}
        \item \texttt{x * 0} to \texttt{0}
        \item \texttt{x * 1} to \texttt{x}
    \end{itemize}
    
    Additionally, properties like commutatity of integer addition can help to factor out common subexpressions.
    
    However, it must also be noted that computations such as floating point arithmetic do not necessarily follow properties satisfied by real numbers, especially involving floating point precision.
\end{definition}

\begin{example}
    Given the code fragment
    \begin{minted}{java}
int x = (2 + z) * j;
int y = (z + 2) * k;
    \end{minted}
    
    Given that \texttt{z} is an integer, meaning \texttt{2 + z} and \texttt{z + 2} are both integer expressions, the two subexpressions are commutative and yield identical results, meaning that it is safe to factor them out to become
    \begin{minted}{java}
int t1 = 2 + z;
int x = t1 * j;
int y = t1 * k;
    \end{minted}
\end{example}

\subsection{Dead Code Elimination}

\begin{definition}[Dead Code Elimination]
    With \textit{control flow analysis} and \textit{data flow analysis}, it can be determined that some parts of the code is never reached and never execute and can safely be removed.
\end{definition}

\begin{example}
    The body of the \texttt{if} statement in the code fragment below is never reached since the condition is always false, then the entire \texttt{if} block can be safety discarded.
    \begin{minted}{java}
int a = 2;
if (a > 3)
{
    System.out.println("Hello World!");
}
    \end{minted}
\end{example}

\subsection{Constant Folding}

\begin{definition}[Constant Folding Optimization]
    Some subexpressions can be eagerly evaluated at compile time to reduce the number of instructions required, especially expressions involving numeric constants.
    
    It is important that semantics must be respectived, especially with respect to precision; that is,
    \begin{itemize}
        \item Compile-time precision must match run-time precision.
    \end{itemize}
\end{definition}

\begin{example}
    Expressions such as
    \begin{minted}{java}
int PI = 3.14159;
int circumference = 2 * PI * r;
    \end{minted}
    
    Can be folded by precomputing the expression \texttt{2 * PI}, giving
    \begin{minted}{java}
int PI = 3.14159;
int circumference = 6.28318 * r;
    \end{minted}
    
    Note that if \texttt{PI} has no later references, it can also be safety discarded by running a Dead Code Elimination pass after the Constant Folding pass.
\end{example}

\subsection{Strength Reduction}

\begin{definition}[Strength Reduction]
    \textit{Strength Reduction} involves substituting expression operations with semantically-equivalent but faster ones.
    
    In the order of slowest to fastest,
    \begin{itemize}
        \item Division, Modulo
        \item Multiplication
        \item Addition, Subtraction, Shifts
        \item Comparisons
    \end{itemize}
    
    That is, divisions and modulo operations are more expensive (slower) than multiplication, and so on.
\end{definition}

\begin{example}
    Expressions such as
    \begin{minted}{java}
5 * x
    \end{minted}
    
    May be substituted with
    \begin{minted}{java}
x << 2 + x
    \end{minted}
    
    Since multiplication is more expensive than bitwise shift and addition.
\end{example}

\begin{definition}[Induction Variables]
    A variable which is modified with each iteration by some common pattern may be subject to strength reduction via \textit{induction}.
\end{definition}

\begin{example}
    In the loop
    \begin{minted}{java}
int i = 10;
int t = 0;
while (i > 0)
{
    i--;
    t = 4 * i;
    a[i] = b[t];
}
    \end{minted}
    
    The expression assigned to $t$ can be substituted with
        \begin{minted}{java}
int i = 10;
int t = 4 * i;
while (i > 0)
{
    i--;
    t = t - 4;
    a[i] = b[t];
}
    \end{minted}
\end{example}

\subsection{Inlining}

\begin{definition}[Inlining]
    \textit{Inlining} intends to avoid function calls which are expensive by replacing the function invocation via copying the function body to the call site and initializing with parameters.
    
    Note that
    \begin{itemize}
        \item Inlined function needs to be sufficiently small, or target code may suffer from code bloat.
        \item Recursive functions need to be handled carefully.
    \end{itemize}
\end{definition}

\begin{example}
    Given the code fragment
    \begin{minted}{c}
#include <stdio.h>

int foo(int a, int b)
{
    return (a + b) * b;
}

int main(void)
{
    int x = 0;
    for (int a = 0; a < 10; a++)
        for (int b = 0; b < 20; b++)
            x += foo(a, b);
            
    printf("%d\n", x);

    return 0;
}
    \end{minted}
    
    Then the call to \texttt{foo(a, b)} can be replaced with its body to give
    \begin{minted}{c}
#include <stdio.h>

int main(void)
{
    int x = 0;
    for (int a = 0; a < 10; a++)
        for (int b = 0; b < 20; b++)
            x += (a + b) * b;
            
    printf("%d\n", x);

    return 0;
}
    \end{minted}
    
    Note that \texttt{foo(int, int)} could be eliminated after an additional Dead Code Elimination pass since it is no longer needed.
\end{example}

\subsection{Loop Optimizations}

\subsubsection{Loop Unrolling}

\begin{definition}[Loop Unrolling]
    \textit{Loop Unrolling} refers to repeat the loop body given a known number of iterations.
    
    \begin{itemize}
        \item Benefits
        \begin{itemize}
            \item Less jumps.
            \item Less condition checking and no need for index variables.
        \end{itemize}
        \item Drawbacks
        \begin{itemize}
            \item Increase in code size.
            \item May be too large for L1 instruction cache.
        \end{itemize}
    \end{itemize}
\end{definition}

\begin{example}
    The loop
    \begin{minted}{c}
for (int i = 0; i < 3; i++)
    printf("Hello World!\n");
    \end{minted}
    
    Could be unrolled to
    \begin{minted}{c}
printf("Hello World!\n");
printf("Hello World!\n");
printf("Hello World!\n");
    \end{minted}
\end{example}

\subsubsection{Fusion / Fission}

\begin{definition}[Loop Fusion / Fission]
    \textit{Loop Fusion} and \textit{Loop Fission} refers to splitting or combining loops with the same index range.
    
    \begin{itemize}
        \item Loop fusion gives less jumps.
        \item Loop fission allows multiple processors to execute each loop.
    \end{itemize}
\end{definition}

\begin{example}
    The following loops demonstrate fusion and fission:
    
    \begin{figure}[H]
        \centering
        \begin{subfigure}{0.45\textwidth}
            \centering
            \begin{minted}{c}
int a[100];
int b[100];
for (int i = 0; i < 100; i++)
{
    a[i] = 1;
    b[i] = 2;
}
            \end{minted}
            \caption{Fusion}
        \end{subfigure}
        \begin{subfigure}{0.45\textwidth}
            \centering
            \begin{minted}{c}
int a[100];
int b[100];
for (int i = 0; i < 100; i++)
{
    a[i] = 1;
}
for (int i = 0; i < 100; i++)
{
    b[i] = 2;
}
            \end{minted}
            \caption{Fission}
        \end{subfigure}
        \caption{Loop Fusion and Loop Fission. Going from left version to right version is \textit{fission}, while going from right version to left version is \textit{fusion}.}
        \label{fig:loop-fusion-fission}
    \end{figure}
\end{example}

\subsubsection{Code Motion}

\begin{definition}[Code Motion]
    \textit{Code Motion} moves loop-invariant (loop-independent) code outside the loop.
    
    \begin{itemize}
        \item Benefits
        \begin{itemize}
            \item Less instructions.
            \item Constants may be stored in registers.
        \end{itemize}
        \item Drawbacks
        \begin{itemize}
            \item Can cause register spill.
        \end{itemize}
    \end{itemize}
\end{definition}

\begin{example}
    For instance, in
    \begin{minted}{c}
#define N 100
while (i <= N - 1) 
{ 
    // ... 
}
    \end{minted}
    
    The predicate check involves computing \texttt{N - 1} each iteration, which is unnecessary and can be moved out to become
    \begin{minted}{c}
#define N 100
int max = N - 1;
while (i <= max) 
{
    // ...
}
    \end{minted}
\end{example}

\subsubsection{Tiling}

\begin{definition}[Tiling]
    \textit{Tiling} creates smaller loops around blocks of data to make sure that the data fits within CPU caches.
\end{definition}

\begin{example}
    For the loop
    \begin{minted}{c}
for (int i = 0; i < N; i++)
{
    // ...
}
    \end{minted}
    
    A nested loop can be introduced which allows finer partitions
    \begin{minted}{c}
for (int i = 0; i < N; i += B)
{
    for (int j = i; j < min(N, j+b); i++) {
        // ...
    }
}
    \end{minted}
\end{example}

\subsubsection{Loop Inversion}

\begin{definition}[Loop Inversion]
    \textit{Inversion} converts a \texttt{while} loop into an equivalent \texttt{do} loop in order to
    \begin{itemize}
        \item Reduce number of jumps, implying fewer pipeline stalls.
    \end{itemize}
\end{definition}

\begin{example}
    For instance, given the \texttt{while} loop
    \begin{minted}{c}
int i = 0;
while (i < 100)
{
    // ...
    i++;
}
    \end{minted}
    
    Is equivalently
    \begin{minted}{c}
int i = 0;
if (i < 100)
{
    do {
        // ...
        i++:
    } while (i < 100);
}
    \end{minted}
\end{example}

\subsubsection{Principles of Locality}

\begin{definition}[Temporal Locality]
    \textit{Temporal Locality} refers to reusing specific data or resource within a short period of time.
\end{definition}

\begin{definition}[Spatial Locality]
    \textit{Spatial Locality} refers to using data between close storage locations, e.g. array elements.
\end{definition}

\subsubsection{Interchange}

\begin{definition}[Interchange]
    Outer and inner loops may be interchanged for better \textit{locality of reference}.
\end{definition}

\begin{example}
    The outer loop performing less iterations could be swapped with the inner loop in
    \begin{minted}{c}
for (int i = 0; i < 10; i++)
    for (int j = 0; j < 20; j++)
        x[i][j] = i + j;
    \end{minted}
    
    To become
    \begin{minted}{c}
for (int j = 0; j < 20; j++)
    for (int i = 0; i < 10; i++)
        x[i][j] = i + j;
    \end{minted}
\end{example}

\subsubsection{Unswitching}

\begin{definition}[Unswitching]
    \textit{Unswitching} extracts conditional inside a loop by copying the loop for true and false branches.
    \begin{itemize}
        \item Helps with parallelism.
        \item Allows optimizing loops independently.
    \end{itemize}
\end{definition}

\begin{example}
    Given the loop
    \begin{minted}{c}
for (int i = 0; i < 100; i++)
{
    doSomething();
    if (predicate)
    {
        doSomethingElse();
    }
}
    \end{minted}
    
    Then it becomes
    \begin{minted}{c}
if (predicate)
{
    for (int i = 0; i < 100; i++)
    {
        doSomething();
        doSomethingElse();   
    }
} 
else 
{
    for (int i = 0; i < 100; i++)
    {
        doSomething();
    }
}
    \end{minted}
\end{example}

\subsubsection{Peephole Optimization vs Super Optimization}

\begin{definition}[Peephole Optimization]
    Aforementioned optimizations are all \textit{peephole} optimizations in that they
    \begin{itemize}
        \item Operate at limited scope.
        \item Use pattern matching against predefined rules.
        \item Keeps trying to apply optimizations until nothing changes.
    \end{itemize}
    
    Note that there may not necessarily be a \textit{correct} order to apply different peephole optimizations, and that one peephole optimization may allow another to function.
\end{definition}

\begin{definition}[Super Optimization]
    Given a program written in machine language, exhaustive search is performed to find the smallest equivalent program.
    \begin{itemize}
        \item Can be used to optimize critical inner loops.
        \item Can be used to find peephole optimizer schemes.
    \end{itemize}
\end{definition}
