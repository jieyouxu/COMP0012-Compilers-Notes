\section{Semantic Analysis}

\begin{definition}[Symbol Table]
    Symbol table contains binding information. I.e. association between identifiers and information such as their scope and type.
\end{definition}

\subsection{CHANGE ME}

\begin{figure}[H]
    \centering
    \fbox{
    \begin{tikzpicture}
        \node [phantomblock] (start) { \phantom{} };
        \node [block, below=1em of start] (lexer) {Lexical Analysis};
        \node [phantomblock, right=0.5em of lexer] {Regular Language};
        \node [block, below=1em of lexer] (parser) {Syntax Analysis};
        \node [phantomblock, right=0.5em of parser] {Context-Free Language};
        \node [block, below=1em of parser] (semantic) {Semantic Analysis};
        \node [phantomblock, right=0.5em of semantic] {Contextual Language};
        \node [phantomblock, below=1em of semantic] (end) {};
        \draw [->] (start) -- (lexer);
        \draw [->] (lexer) -- (parser);
        \draw [->] (parser) -- (semantic);
        \draw [->] (semantic) -- (end);
    \end{tikzpicture}
    }
    \caption{Languages used for different phases}
    \label{fig:language-in-different-phases}
\end{figure}

\begin{figure}
    \centering
    \begin{tikzpicture}
        %\draw (1,1)--(0,0)
        
    \end{tikzpicture}
    
\end{figure}