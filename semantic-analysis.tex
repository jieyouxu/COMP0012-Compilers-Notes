\section{Semantic Analysis}

\subsection{Overview}

\begin{figure}[H]
    \centering
    \fbox{
    \begin{tikzpicture}
        \node [phantomblock]                            (start)     { \phantom{} };
        \node [block, below=1em of start]               (lexer)     {Lexical Analysis};
        \node [phantomblock, right=0.5em of lexer]                  {Regular Language};
        \node [block, below=1em of lexer]               (parser)    {Syntax Analysis};
        \node [phantomblock, right=0.5em of parser]                 {Context-Free Language};
        \node [block, below=1em of parser]              (semantic)  {Semantic Analysis};
        \node [phantomblock, right=0.5em of semantic]               {Contextual Language};
        \node [phantomblock, below=1em of semantic]     (end)       {};
        \draw [->] (start)      -- (lexer);
        \draw [->] (lexer)      -- (parser);
        \draw [->] (parser)     -- (semantic);
        \draw [->] (semantic)   -- (end);
    \end{tikzpicture}
    }
    \caption{Languages used for different phases}
    \label{fig:language-in-different-phases}
\end{figure}

\begin{itemize}
    \item \textit{Semantics} (or \textit{context}) consists of \textit{scope} and \textit{type}.
    
    \item The reason why we cannot check \textit{semantics} in parsing phase is due to the limitation of Context-Free Grammar. \textit{Semantics}-checking requires surrounding context which Context-Free Grammar lacks.
    
    \item So the way to do semantics checking is to check outside the grammar and try to find contextual mistakes if there is any.
\end{itemize}


\subsection{Lexical/Static scope}
\begin{definition}[Declaration]
    Declaration is a syntactic construct that associates information with a name.
\end{definition}


\begin{definition}[Scope rules]
    Scope rules of a language determine which declaration applies to a name when the name appears in the text of a program.
\end{definition}


\begin{definition}[Binding]
    Binding is an association between two things, such as a name and the thing it represents.
\end{definition}

\begin{definition}[Scope of identifier]
    The scope of a identifier is the part of a program in which it can be used to find its referred identity.
\end{definition}

\begin{definition}[\enquote{Hole} in a binding's scope]
    A binding has a hole in its scope when it is hidden by a nested declaration of the same name.
\end{definition}

\begin{definition}[Qualifier]
    In some languages, qualifier allows access to the outer meaning of an identifier. 
\end{definition}


\subsection{Dynamic scope}

\begin{definition}[Dynamic scope rules]
    resolve the reference according to the most recent declaration w.r.t. the runtime flow of execution
\end{definition}


\subsection{Referencing Environment}

\begin{definition}[Referencing environment]
    Referencing environment is the sequence of scopes that is to be examined to find the current binding of a given name.
\end{definition}


