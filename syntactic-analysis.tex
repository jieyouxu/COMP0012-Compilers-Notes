\section{Syntactic Analysis}

The language used in \textit{Syntactic Analysis} is the \textit{Context-Free Grammar} (CFG). Sometimes \textit{Parsing Expression Grammar} (PEG) is also used, but PEG is not covered in this course.

\subsection{Context-Free Grammar}

\begin{definition}[Context-Free Grammar (CFG)]
    A \textit{Context-Free Grammar} (CFG) $G$ can be described by the 4-tuple
    \begin{equation}
        G = \langle
            \mathcal{N},
            \mathcal{T},
            S,
            \mathcal{R}
        \rangle
    \end{equation}
    
    Where
    \begin{enumerate}
        \item \textbf{Non-terminals}: $\mathcal{N}$ is the finite set of \textit{non-terminals} (uppercase by convention).
        \item \textbf{Terminals}: $\mathcal{T}$ is the finite set of \textit{terminals} (lowercase by convention).
        \item \textbf{Start Symbol}: $S \in \mathcal{T}$ is the \textit{start} symbol.
        \item \textbf{Production Rules}: $\mathcal{R} \coloneqq \mathcal{N} \to (\mathcal{N} \cup \mathcal{T})^\ast$ is the set of finite relations, termed \textit{productions} or \textit{rules} of the grammar.
    \end{enumerate}
\end{definition}

\begin{remark}
    By convention,
    \begin{itemize}
        \item Nonterminals are written in uppercase.
        \item Terminals are either punctuation characters or written in lowercase.
        \item Start symbol is the left-hand side non-terminal of the first product.
    \end{itemize}
\end{remark}

\begin{example}
    A \textit{production}
    \begin{equation*}
        X \to Y_1 \cdots Y_n
    \end{equation*}
    Means that the non-terminal $X$ can be \textit{replaced} by $Y_1 \cdots Y_n$. Equivalently, the production right-hand side $Y_1 \cdots Y_n$ is \textit{produced} by $X$.
\end{example}

\subsubsection{Language of a Context-Free Grammar}

\begin{definition}[Language of a Context-Free Grammar]
    Given context-free grammar $G$ with the start symbol $S$, the \textit{language} of $G$ is
    \begin{equation}
        L(G) \coloneqq \left\lbrace a_1 \cdots a_n \mid S \xrightarrow{\ast} a_1 \cdots a_n, a_i \in \mathcal{T} \right\rbrace
    \end{equation}
    
    That is, $L(G)$ is the set of all strings of \textit{terminals} for which the grammar $G$ can generate in zero or more steps.
\end{definition}

\subsubsection{Terminals}

\begin{definition}[Terminals]
    Characters in the input alphabet $\mathcal{T}$ are called \textit{terminals} because there does not exist any production rules which can replace them.
    
    That is, \textit{terminals} only appear in the \textit{right-hand side} of any production rule.
\end{definition}

\begin{remark}
    If terminals are generated by any replacement steps, then they are \textit{permanent}.
    
    \textit{Terminals} are often \textit{lexemes} in the language. For example, the terminal \texttt{void} is a keyword lexeme in the C language.
\end{remark}

\subsection{Right Regular Grammar}

\begin{definition}[Right-regular Grammar]
    A \textit{Right-regular Grammar} $G$ is the 4-tuple
    \begin{equation}
        G = \langle
            \mathcal{N},
            \mathcal{T},
            S,
            \mathcal{R}
        \rangle
    \end{equation}
    
    And also satisfying three additional \textit{constraints}; given $A, B \in \mathcal{N}$ and $a \in \mathcal{T}$,
    \begin{enumerate}
        \item $A \to a$
        \item $A \to aB$
        \item $A \to \epsilon$
    \end{enumerate}
\end{definition}

\begin{remark}
    Note that a \textit{Left-regular Grammar} has the above definition with only one difference, in the second constraint, for
    \begin{equation}
        A \to Ba
    \end{equation}

    If left and right rules are mixed together, a linear grammar is generated which is context-free.
\end{remark}

\subsubsection{Chomsky Hierarchy}

\begin{definition}[Chomsky Hierarchy]
    The \textit{Chomsky Hierarchy} classifies different languages in increasing expressive power.
    
    \begin{figure}[H]
        \centering
        \begin{tabularx}{\textwidth}{@{} X X X X @{}}
            \toprule
            Expressiveness & Grammar & Production Constraint & Automata \\
            \midrule
            $\uparrow$ & Universal & $\alpha \to \beta$ & Turing \par Machine \\
            \phantom{} & Context-Sensitive & $\alpha A \beta \to \alpha \delta \beta$ & Linear \par Bounded \par Automata \\
            \phantom{} & Context-Free & $A \to \alpha$ & Push-Down \par Automata \\
            $\downarrow$ & Regular (right) & $A \to a \mid aB \mid \epsilon$ & Deterministic \par Finite \par Automata \\
            \bottomrule
        \end{tabularx}
        \caption{Chomsky Hierarchy}
        \label{fig:chomsky-hierarachy}
    \end{figure}
\end{definition}


